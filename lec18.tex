\documentclass[twoside]{article}

% We add packages, macros here:
%!TEX root =  lec-template.tex
\usepackage{lmodern}
\usepackage[english]{babel}
\usepackage{latexsym}
\usepackage{amsmath}
\usepackage{mathrsfs}
\usepackage{amssymb}
\usepackage{mathtools}
\usepackage[inline,shortlabels]{enumitem} % we prefer enumitem because of its margin adjustment caps
\usepackage{bm}
\usepackage{datetime}
\usepackage[table,xcdraw]{xcolor}
\usepackage{accents}
\usepackage{tikz}
\usepackage{listings}
\usepackage{mdframed}
\usepackage{pgfplots}
\usepackage{pgfplotstable}
\usepackage[boxed]{algorithm}
\usepackage{algpseudocode}
\usepackage{dsfont}
\usepackage{color}
\usepackage{colortbl}
\usepackage{pifont}
\usepackage[bf,font=small,singlelinecheck=off]{caption}
\usepackage{microtype} % improved spacing between words for easier reading
\usepackage{float}
\usepackage{xfrac} % sfrac
\usepackage{xspace}

\linespread{1.1}


\usepackage[textsize=tiny]{todonotes}
\makeatletter
\renewcommand{\todo}[2][]{\@todo[#1]{#2}}
\makeatother

\setlength{\marginparwidth}{10ex}
\newcommand{\todoc}[2][]{\todo[size=\scriptsize,color=blue!20!white,#1]{Csaba: #2}}
\newcommand{\todoj}[2][]{\todo[size=\scriptsize,color=red!20!white,#1]{Jincheng: #2}}

\usepackage{hyperref}
\hypersetup{
    unicode=false,          % non-Latin characters in Acrobat�s bookmarks
    pdftoolbar=true,        % show Acrobat�s toolbar?
    pdfmenubar=true,        % show Acrobat�s menu?
    pdffitwindow=false,     % window fit to page when opened
    pdfstartview={FitH},    % fits the width of the page to the window
    pdftitle={},    % title
    pdfauthor={},     % author
    pdfsubject={Theory, Machine Learning, Lectures},   % subject of the document
    pdfcreator={},   % creator of the document
    pdfproducer={}, % producer of the document
    pdfkeywords={theory} {machine learning} {lecture notes} {CMPUT 654} {Fall 2023}, % list of keywords
    pdfnewwindow=true,      % links in new window
    colorlinks=true,       % false: boxed links; true: colored links
    linkcolor=blue,          % color of internal links (change box color with linkbordercolor)
    citecolor=blue,        % color of links to bibliography
    filecolor=magenta,      % color of file links
    urlcolor=cyan           % color of external links
}
\usepackage{amsthm}
\usepackage{times}
\usepackage{natbib}
\usepackage{nicefrac}
\usepackage{wrapfig}
\usepackage[capitalize]{cleveref}
\usepackage[nottoc,numbib]{tocbibind}

%\usepackage[bmargin=0.75in]{geometry}
\usepackage[margin=1.1in]{geometry}
\usepackage[normalem]{ulem}


%%%%%%%%%%%%%%%%%%%%%%%%%%%%%%%%
% HYPHENATION
%%%%%%%%%%%%%%%%%%%%%%%%%%%%%%%%

\pretolerance=5000
\tolerance=9000
\emergencystretch=0pt
\righthyphenmin=4
\lefthyphenmin=4


\bibliographystyle{plainnat}

\newcommand{\E}{\mathbb{E}}
\newcommand{\EE}[1]{\E[#1]}
\newcommand{\PP}{\mathbb{P}}
\newcommand{\Prob}[1]{\mathbb{P}(#1)}
\newcommand{\one}[1]{\mathbb{I}\{#1\}}
\newcommand{\Supp}{\operatorname{supp}}
\newcommand{\ip}[1]{\langle #1 \rangle}
\newcommand{\bip}[1]{\left\langle #1 \right\rangle}
\newcommand{\norm}[1]{\|#1\|}
\newcommand{\R}{\mathbb{R}}
\newcommand{\N}{\mathbb{N}}
\newcommand{\cA}{\mathcal{A}}
\newcommand{\cB}{\mathcal{B}}
\newcommand{\cC}{\mathcal{C}}
\newcommand{\cD}{\mathcal{D}}
\newcommand{\cE}{\mathcal{E}}
\newcommand{\cF}{\mathcal{F}}
\newcommand{\cG}{\mathcal{G}}
\newcommand{\cH}{\mathcal{H}}
\newcommand{\cK}{\mathcal{K}}
\newcommand{\cL}{\mathcal{L}}
\newcommand{\cN}{\mathcal{N}}
\newcommand{\cP}{\mathcal{P}}
\newcommand{\cQ}{\mathcal{Q}}
\newcommand{\cR}{\mathcal{R}}
\newcommand{\cS}{\mathcal{S}}
\newcommand{\sA}{\mathscr A}

\newcommand{\cM}{\mathcal{M}}
\newcommand{\cX}{\mathcal{X}}
\newcommand{\cY}{\mathcal{Y}}
\newcommand{\NN}{\mathbb{N}}
\newcommand{\RR}{\mathbb{R}}
\newcommand{\VV}[1]{\mathbb{V}[#1]}

\newcommand{\epsapp}{\epsilon}
\newcommand{\epssub}{\delta}

\DeclareMathOperator{\Range}{range}
\newcommand{\rows}{\operatorname{rows}}

\renewcommand{\epsilon}{\varepsilon}
\newcommand{\ceil}[1]{\left\lceil {#1} \right\rceil}
\newcommand{\floor}[1]{\left\lfloor {#1} \right\rfloor}
\newcommand{\ones}{\mathbf{1}}
\newcommand{\zeros}{\mathbf{0}}
\DeclareMathOperator*{\argmin}{arg\ min}
\DeclareMathOperator*{\argmax}{arg\ max}


\def\rvzero{{\mathbf{0}}}
\def\rvone{{\mathbf{1}}}

\def\identiymatrix{\mathbf{Id}}

\newcommand{\softmax}{\mathrm{softmax}}
\newcommand{\KL}{D_{\mathrm{KL}}}

% Graph
\def\gA{{\mathcal{A}}}
\def\gB{{\mathcal{B}}}
\def\gC{{\mathcal{C}}}
\def\gD{{\mathcal{D}}}
\def\gE{{\mathcal{E}}}
\def\gF{{\mathcal{F}}}
\def\gG{{\mathcal{G}}}
\def\gH{{\mathcal{H}}}
\def\gI{{\mathcal{I}}}
\def\gJ{{\mathcal{J}}}
\def\gK{{\mathcal{K}}}
\def\gL{{\mathcal{L}}}
\def\gM{{\mathcal{M}}}
\def\gN{{\mathcal{N}}}
\def\gO{{\mathcal{O}}}
\def\gP{{\mathcal{P}}}
\def\gQ{{\mathcal{Q}}}
\def\gR{{\mathcal{R}}}
\def\gS{{\mathcal{S}}}
\def\gT{{\mathcal{T}}}
\def\gU{{\mathcal{U}}}
\def\gV{{\mathcal{V}}}
\def\gW{{\mathcal{W}}}
\def\gX{{\mathcal{X}}}
\def\gY{{\mathcal{Y}}}
\def\gZ{{\mathcal{Z}}}

% Sets
\def\sA{{\mathbb{A}}}
\def\sB{{\mathbb{B}}}
\def\sC{{\mathbb{C}}}
\def\sD{{\mathbb{D}}}
% Don't use a set called E, because this would be the same as our symbol
% for expectation.
\def\sE{{\mathbb{E}}}
\def\sF{{\mathbb{F}}}
\def\sG{{\mathbb{G}}}
\def\sH{{\mathbb{H}}}
\def\sI{{\mathbb{I}}}
\def\sJ{{\mathbb{J}}}
\def\sK{{\mathbb{K}}}
\def\sL{{\mathbb{L}}}
\def\sM{{\mathbb{M}}}
\def\sN{{\mathbb{N}}}
\def\sO{{\mathbb{O}}}
\def\sP{{\mathbb{P}}}
\def\sQ{{\mathbb{Q}}}
\def\sR{{\mathbb{R}}}
\def\sS{{\mathbb{S}}}
\def\sT{{\mathbb{T}}}
\def\sU{{\mathbb{U}}}
\def\sV{{\mathbb{V}}}
\def\sW{{\mathbb{W}}}
\def\sX{{\mathbb{X}}}
\def\sY{{\mathbb{Y}}}
\def\sZ{{\mathbb{Z}}}

\newcommand{\dimE}{\mathrm{dim}_{\mathcal{E}}}
\DeclareMathOperator{\diam}{diam}


%%
%% ADD PACKAGES here:
%%
%
%\usepackage{amsmath,amsfonts,graphicx}
%
%
% The following commands set up the lecnum (lecture number)
% counter and make various numbering schemes work relative
% to the lecture number.
%

\newcounter{lecnum}
\renewcommand{\thepage}{\thelecnum-\arabic{page}}
\renewcommand{\thesection}{\thelecnum.\arabic{section}}
\renewcommand{\theequation}{\thelecnum.\arabic{equation}}
\renewcommand{\thefigure}{\thelecnum.\arabic{figure}}
\renewcommand{\thetable}{\thelecnum.\arabic{table}}


%
% The following macro is used to generate the header.
%
\newcommand{\lecture}[4]{
   \pagestyle{myheadings}
   \thispagestyle{plain}
   \newpage
   \setcounter{lecnum}{#1}
   \setcounter{page}{1}
   \noindent
   \begin{center}
   \framebox{
      \vbox{\vspace{2mm}
    \hbox to 6.28in { {\bf CMPUT 654 Fa 23: Theoretical Foundations of Machine Learning \hfill Fall 2023} }
       \vspace{4mm}
       \hbox to 6.28in { {\Large \hfill Lecture #1: #2  \hfill} }
       \vspace{2mm}
       \hbox to 6.28in { {\it Lecturer: #3 \hfill Scribes: #4} }
      \vspace{2mm}}
   }
   \end{center}
   \markboth{Lecture #1: #2}{Lecture #1: #2}

   \noindent {\bf Note}: {\it 
   \LaTeX\  template courtesy of UC Berkeley EECS dept. (\href{https://inst.eecs.berkeley.edu/~cs294-8/sp03/Materials/}{link} to directory)
   }

   \noindent {\bf Disclaimer}: {\it These notes have \underline{\textbf{not}} been subjected to the
   usual scrutiny reserved for formal publications. They may be
   distributed outside this class only with the permission of the
   Instructor.} \vspace*{4mm}
}
%
% Convention for citations is authors' initials followed by the year.
% For example, to cite a paper by Leighton and Maggs you would type
% \cite{LM89}, and to cite a paper by Strassen you would type \cite{S69}.
%%%%%%%%% (To avoid bibliography problems, for now we redefine the \cite command.)
%%%%%%%%% Also commands that create a suitable format for the reference list.
%%%%%%%%\renewcommand{\cite}[1]{[#1]}
%%%%%%%%\def\beginrefs{\begin{list}%
%%%%%%%%        {[\arabic{equation}]}{\usecounter{equation}
%%%%%%%%         \setlength{\leftmargin}{2.0truecm}\setlength{\labelsep}{0.4truecm}%
%%%%%%%%         \setlength{\labelwidth}{1.6truecm}}}
%%%%%%%%\def\endrefs{\end{list}}
%%%%%%%%\def\bibentry#1{\item[\hbox{[#1]}]}

%Use this command for a figure; it puts a figure in wherever you want it.
%usage: \fig{NUMBER}{SPACE-IN-INCHES}{CAPTION}
\newcommand{\fig}[3]{
			\vspace{#2}
			\begin{center}
			Figure \thelecnum.#1:~#3
			\end{center}
	}
% Use these for theorems, lemmas, proofs, etc.

%!TEX root =  lec-template.tex
%%%%%%%%%%%%%%%%%%%%%%%%%%%%%%%%
% THEOREMS
%%%%%%%%%%%%%%%%%%%%%%%%%%%%%%%%
\theoremstyle{plain}
\newtheorem{theorem}{Theorem}[lecnum]
\newtheorem{claim}[theorem]{Claim}
\newtheorem{proposition}[theorem]{Proposition}
\newtheorem{lemma}[theorem]{Lemma}
\newtheorem{corollary}[theorem]{Corollary}
\newtheorem{example}[theorem]{Example}
\theoremstyle{definition}
\newtheorem{definition}[theorem]{Definition}
\newtheorem{assumption}[theorem]{Assumption}
\newtheorem{remark}[theorem]{Remark}
\newtheorem{exercise}[theorem]{Exercise}
\theoremstyle{remark}


%\newtheorem{theorem}{Theorem}[lecnum]
%\newtheorem{lemma}[theorem]{Lemma}
%\newtheorem{proposition}[theorem]{Proposition}
%\newtheorem{claim}[theorem]{Claim}
%\newtheorem{corollary}[theorem]{Corollary}
%\newtheorem{definition}[theorem]{Definition}
%\newenvironment{proof}{{\bf Proof:}}{\hfill\rule{2mm}{2mm}}

% **** IF YOU WANT TO DEFINE ADDITIONAL MACROS FOR YOURSELF, PUT THEM HERE:

\newcommand\myE{\mathbb{E}}

\begin{document}
%FILL IN THE RIGHT INFO.
%\lecture{**LECTURE-NUMBER**}{**DATE**}{**LECTURER**}{**SCRIBE**}
\lecture{18}{November 2}{Csaba Szepesv\'ari}{Vedant Vyas}
%\footnotetext{These notes are partially based on those of Nigel Mansell.}

% **** YOUR NOTES GO HERE:

% Some general latex examples and examples making use of the
% macros follow.  
%**** IN GENERAL, BE BRIEF. LONG SCRIBE NOTES, NO MATTER HOW WELL WRITTEN,
%**** ARE NEVER READ BY ANYBODY.
\section{Leave One Out (L.O.O) Stability}

\begin{definition}
$k: \overset{\text{target}}{Z^n} \cup \overset{\text{auxillary}}{Z^{n+1}} \rightarrow \mathcal{M}_1(\mathbb{R}^{\mathbb{Z}}) \text{ is $\epsilon$-LOO (Leave One Out) stable with $\epsilon:\mathbb{Z}^{n+1}$ } \cup \mathbb{Z} \rightarrow [0, \infty) \\ \text{ if } \hspace{0.1cm} \forall z_{1:n+1} \in Z^{n+1} \mu_k(z_{1:n}, z_{n+1} ) \le \mu_k(z_{1:n+1}, z_{n+1}) + \epsilon(z_{1:n+1}, z_{n+1})$
\end{definition}
\begin{proposition}
Assume 
$\mu_k(z_{1:n+1}^{i \leftrightarrow n+1}, z_{n+1}) = \mu_k(z_{1:n+1}, z_{n+1}) \text{ and }
\epsilon(z_{1:n+1}^{i\leftrightarrow n+1}, z_{n+1}) = \epsilon(z_{1:n+1}, z_{n+1}) \\ \& \text{ K is $\epsilon$-LOO stable then  } $

$$
\mathbb{E}\left[\mu_{k}\left(z_{1: n}, z_{n+1}\right)\right] \leq \mathbb{E} P_{n+1} \mu_{k}\left(z_{1: n+1}\right)+\mathbb{E} P_{n+1} \varepsilon\left(z_{1: n+1}\right)
$$

\end{proposition}
\noindent \textbf{Note: }
From our assumptions on $\mu_k$ we can get complete symmetry as to swap i $\leftrightarrow$ j, we can do $i \leftrightarrow n+1$, and then $n+1 \leftrightarrow j$

\begin{proof}
\begin{align*}
z_{1:n+1}^{i \leftrightarrow n+1} &= (z_1,,,,z_{i-1},z_{n+1},....z_i) \\
(z_{1:n+1}^{i \leftrightarrow n+1}, z_i) &\stackrel{P}{=}(z_{1:n+1}, z_{n+1}) \quad \quad \text{(Same joint distribution)} \\
(\mu_k + \epsilon)(z_{1:n+1}^{i \leftrightarrow n+1}, z_i) &\sim (\mu_k + \epsilon)(z_{1:n+1}, z_{n+1}) \\
\text{So, } \mu_k(z_{1:n}, z_{n+1}) &\le (\mu_k + \epsilon) (z_{1:n+1}, z_{n+1}) \\
&\stackrel{P}{=} (\mu_k + \epsilon) (z_{1:n+1}^{i \leftrightarrow n+1}, z_{i}) \\
\end{align*}
Taking expectation on both sides, and averaging over i would lead to the desired result
\end{proof}

\section{First Order Optimality}
\begin{lemma}

$f : C \rightarrow \mathbb{R}, C \subseteq \mathbb{R}^{d}$ closed, convex, $C \neq \varnothing$

$$
x^{*} \in \underset{x \in C}{\operatorname{argmin}} f(x) \text {. }
$$

% \begin{center}
% \includegraphics[max width=\textwidth]{2023_11_18_7430c45718e3a62a4a59g-3}
% \end{center}

(1.) $\exists \theta \in \partial f\left(x^{*}\right)$ s.t. $\theta^{T}\left(x-x^{*}\right) \geq 0$.

(2.) Assume $f \text{ is } \lambda-S O C, \tilde{\tau} \in \partial f(x), \tilde{\theta}^{\top}\left(x^{*}-x\right) \geq-g\left\|x-x^{*}\right\|$ for some $g>0 \Rightarrow\left\|x-x^{*}\right\| \leq \frac{g}{\lambda}$.
\end{lemma}

\begin{proof}
\begin{equation*}
    f'(x^*; x-x^*) = \theta^\top (x-x^*) \text{ for some $\theta \in  \partial f(x)$}
\end{equation*}

\begin{align*}
    f(x^*) &\geqslant f(x) + \tilde{\theta}^\top (x^* - x) + \frac{\lambda}{2} \|x^* - x\|^2 \\
    &\geqslant f(x) - g \|x - x^*\| + \frac{\lambda}{2} \|x^* - x\|^2
\end{align*}

\begin{align*}
    f(x) &\geq f(x^*) + \theta^\top (x - x^*) + \frac{\lambda}{2} \|x - x^*\|^2 \\
    &\geqslant f(x^*) + \frac{\lambda}{2} \|x - x^*\|^2 \hspace{1cm} \text{(Using 1)}
\end{align*}

\begin{equation*}
    g \|x - x_*\| \geqslant \lambda \|x - x^*\|^2
\end{equation*}

\hspace{6cm} \text{Q.E.D.}

\end{proof}

\begin{theorem}

$G=\left\{g_{w}: w \in C\right\}, C \subseteq \mathbb{R}^{d}$ closed \text{ and } convex. 
$w \mapsto g_{w}(z) \quad\left(W_{1}, \|\cdot\|_{2}\right) \rightarrow \left(\mathbb{R} ,  \mid \cdot \mid \right) \quad G(z)$-Lipschitz \\ $h: \mathbb{R}^{d} \rightarrow \mathbb{R} ; \quad \bar{g}_{w}(z)=g_{w}(z)+h(w)$.

\noindent Assume $\quad \omega \mapsto P_{z_{1: n}} \bar{g}_{w}$ is $\lambda$-SOC ( $\lambda$ Strongly Convex) ; $\forall z_{1: n} \in Z^{n}$

$$
\begin{aligned}
& \cA \left(z_{1: n}\right):=\underset{w \in C}{\operatorname{argmin}} P_{z_{1:n}} \bar{g}_{w}=\underset{w \in C}{\operatorname{argmin}} P_{z_{1:n}} g_{w}+h(w) \\
& \cA \left(z_{1: n+1}\right)=\underset{w \in C}{\operatorname{argmin}} P_{z_{1: n+1}} g_{w}+\frac{n}{n+1} h(w)
\end{aligned}
$$

Then: 
1.) $\cA$ is $\varepsilon\left(z_{1: n+1}, z_{n+1}^{\prime}\right)=\frac{G\left(z_{n}^{\prime}+1\right)^{2}}{\lambda(n+1)}-$LOO Stable

2.) $\cA$ is $\frac{2\|G\|_{\infty}^{2}}{\lambda n}$ uniformly stable

3.) $z_{1: n} \sim p^{\otimes n}$. Then, for $w_{n}=\cA\left(z_{1: n}\right)$,

$$
\mathbb{E} P g_{w_{n}} \leqslant \operatorname{iuf}_{\omega \in C}\left(P g_{w}+h(\omega)\right)+\frac{\mathbb{E} G^2\left(z_{1}\right)}{\lambda(n+1)}
$$

\end{theorem}
\begin{proof}
\noindent\mbox{}\\
\noindent\textbf{Part 1}: L.O.O Stability \\

$$
\begin{aligned}
\text{Let } Z_{1:n+1} &\sim P^{\otimes n+1} \\
L_n(w) &= P_{z_{1: n}} \bar{g}_{w} \\
w_{n} &= \cA \left(z_{1: n}\right) = \operatorname{argmin}_{w} \, L_n(w)  \\
L_{n+1}(w) &= P_{z_{1: n+1}} g + \frac{n}{n+1} h(w) \\
w_{n+1} &= \cA \left(z_{1: n+1}\right) = \operatorname{argmin}_{w} \, L_{n+1}(w)
\end{aligned}
$$
Assume $L_{n}, L_{n+1}$ are differentiable. By F.O.O. Lemma (18.4): $\theta_{n}:=\nabla L_{n}\left(w_{n}\right)$ s.t.

$$
\theta_{n}^{\top}\left(w_{n+1}-w_{n}\right) \geqslant 0 . \quad \text{(*)}
$$

$$
\begin{aligned}
& \text{Now,} \, L_{n+1}(w) = \frac{n}{n+1} L_{n}(w) + \frac{1}{n+1} g_{w}\left(z_{n+1}\right), \\
& \nabla L_{n+1}\left(w_{n}\right) = \frac{n}{n+1} \nabla L_{n}\left(w_{n}\right) + \frac{1}{n+1} \nabla g_{w_{n}}\left(z_{n+1}\right), \\
& \Rightarrow \nabla L_{n+1}\left(w_{n}\right)^{T}\left(w_{n+1} - w_{n}\right) \stackrel{(*)}{\geqslant} \frac{1}{n+1} \nabla g_{w_{n}}\left(z_{n+1}\right)^{\top}\left(w_{n+1} - w_{n}\right), \\
& \geqslant -\frac{G\left(z_{n+1}\right)}{n+1}\left\|w_{n+1} - w_{n}\right\|, \\
& \downarrow \\
& \left\|\nabla g_{w}\right\| \leqslant G, \, \hspace{1cm}\text{(Cauchy Schwartz)}.
\end{aligned}
$$
Now by using F.O.O Lemma (Part 2), we get
$$
\begin{aligned}
& \Rightarrow\left\|W_{n+1}-W_{n}\right\| \leq \frac{G\left(z_{n+1}\right)}{\lambda(n+1)} \\
& \Rightarrow g_{w_{n}}\left(z_{n+1}\right)-g_{w_{n+1}}\left(z_{n+1}\right) \leq G\left(z_{n+1}\right)\left\|w_{n}-w_{n+1}\right\| \\
& \leq \frac{G^{2}\left(z_{n+1}\right)}{\lambda(n+1)} \\
& \Rightarrow \cA \text { is } \frac{Q^{2}}{\lambda(t+1)}-L \cdot O.O \text { stable }
\end{aligned}
$$
\textbf{Part 2}: Uniform Stability $\quad$ Proof for Uniform stability follows similary but with the use of $\left\|\cdot \right\|_\infty$ \\


\noindent \textbf{Part 3}: Symmetry \\

$$
\begin{aligned}
\cA \left(z_{1: n+1}^{i \leftrightarrow n+1}\right) &= \cA \left(z_{1: n+1}\right), \\
\varepsilon\left(z_{1: n+1}^{i \oplus n+1}\right) &= \varepsilon\left(z_{1}: n-1\right) .
\end{aligned}
$$

LOO Theorem: $\quad Z_{1: n} \sim P^{\otimes n} ; W_{n}: A\left(Z_{1: n}\right), W_{n+1}=A\left(Z_{1: n+1}\right)$.

$$
\begin{aligned}
\mathbb{E} P g_{w_{n}} & \leq \mathbb{E} P_{n+1} g_{w_{n+1}}+\mathbb{E} P_{n+1} \varepsilon\left(z_{n: n+1}\right) \\
& \leq \mathbb{E} L_{n+1}\left(w_{n+1}\right)+\frac{\mathbb{E} G^{2}\left(z_{1}\right)}{\lambda(n+1)} \\
& \leq \mathbb{E} L_{n+1}(\omega)+\frac{1}{n+1} h(\omega)+\frac{\mathbb{E} G^{2}\left(z_{1}\right)}{\lambda(n+1)} \\
& =P g_{w}+h(\omega) +\frac{\mathbb{E} G^{2}\left(z_{1}\right)}{\lambda(n+1)}
\end{aligned}
$$

\end{proof}
\newpage 

\text{Example)}
$$
\begin{aligned}
& \overset{hinge loss} {l\left(f_{1}(x, y)\right)}=\max (1-\tilde{f(x) y}, 0), y \in {\pm 1} \\
& f_{w}(\alpha)=w^{\top} \psi(x), w \in \mathbb{R}^{d} \\
& g(w)=\frac{\lambda}{2}\|w\|^{2} ; \quad h=0 .
\end{aligned}
$$

$$
\begin{aligned}
\text { Then } \mathbb{E} \left[ P \ell \circ f_{w_{n}} \right] 
 &\leq \inf _{w} P l \circ f_{w}+\frac{\lambda}{2}\|w\|^{2}+ \frac{\mathbb{E}\left( \left\| \psi(X) \right\| + \sqrt{2\lambda} \right)^2}{\lambda (n+1)}
\\
\end{aligned}
$$

\begin{proof}

\vspace{5mm}
Consider the function defined by
\begin{equation}
g_{\omega}(x, y) = \ell\left(f_{w_{1}}(x, y)\right) + \frac{\lambda}{2}\|\omega\|^2.
\end{equation}
The mapping $\omega \mapsto g_{w}(z)$ induces a gradient
\begin{equation}
\nabla g_{w(1)} = \underbrace{l^{\prime}(f w, z)}_{\in\{0,1\}} \psi + \underbrace{\lambda \omega}_{\text{unbounded}}.
\end{equation}
Given $z_{1: n} \in Z^{n}$ and $w_{n} \in \operatorname{argmin} L_{n}(w)$, we have
\begin{equation}
\frac{\lambda}{2}\left\|w_{n}\right\|^2 \leq L_n\left(w_{n}\right) \leq L_n(0) \leq 1 \Rightarrow \left\|w_{n}\right\| \leq \sqrt{\frac{2}{\lambda}},
\end{equation}
where
\begin{equation}
\cA \left(z_{1:n}\right) = \operatorname{argmin}_{w \in C} L_n(w) \quad \text{and} \quad C = \left\{w : \left\|w\right\| \leq \sqrt{\frac{2}{\lambda}}\right\}.
\end{equation}
\end{proof}



\begin{theorem}
Assume $\forall z \in Z$, the mapping $w \mapsto g_{w}(z)$ is $\lambda$-strongly convex and $L$-smooth. Then the following properties hold:
\begin{align}
    \cA \left(z_{1: n}\right) &= \underset{w}{\operatorname{argmin}}\, P_{z_{1:n}} g_{w}, \\
    \cA \left(z_{1: n+1}\right) &= \operatorname{argmin}_{w}\, P_{z_{1: n+1}} g_{w}, \\
    \varepsilon\left(z_{1: n+1}, z_{n+1}'\right) &= \left(1+\frac{L}{2\lambda n}\right) \frac{\|\nabla g_{\cA \left(z_{1: m}\right)}\left(z_{n+1}'\right)\|_{2}^{2}}{\lambda n}.
\end{align}
\begin{enumerate}
    \item $\cA$ is $\varepsilon$-L.O.O stable.
    \item If $L \leq 0.2 \lambda n$, then
    \begin{equation}
        \mathbb{E} P g_{\omega_{n}} \leq \inf_{\omega} \left(P g_{\omega} + \frac{2.2}{\lambda n} P\|\nabla g_{\omega}\|_{2}^{2}\right).
    \end{equation}
\end{enumerate}
\end{theorem}

\textbf{Example:} 
let the loss function $l$ and regularization term $g$ be defined as:
\begin{align*}
l\left(f_{1}(x, y)\right) &= (f(x)-y)^{2}, \\
g(w) &= \frac{\lambda}{2}\|w\|_{2}^{2}.
\end{align*}

Then, the model $f$ parameterized by weights $w$, and the composite objective function $g_{\omega}$ can be expressed as:
\begin{align*}
f &= f_{w} = w^{\top} \psi, \\
g_{\omega}(z) &= l\left(f_{w}, z\right) + g(\omega), \text{ implying strong convexity } (\lambda-\text{SOC}). \\
\nabla g_{w}(z) &= 2\left(f_{w}(x) - y\right) \psi(x) + \lambda \omega, \\
\nabla^{2} g_{w}(z) &= 2 \psi(x)^{\top} \psi(x) + \lambda I, \\
\lambda_{\max}\left(\nabla^{2} g_{\omega}(z)\right) &\leq 2\|\psi(x)\|^{2} + \lambda.
\end{align*}

We define the Lipschitz constant $L$ as:
\begin{align*}
L &:= \sup_{x} \left(2\|\psi(x)\|^{2} + \lambda\right).
\end{align*}

If $L \leqslant 0.2 \lambda n$, it follows that:
\begin{equation*}
\mathbb{E} P g_{w_{n}} \leq \inf_{\omega} \left(P g_{\omega} + \frac{8.8}{\lambda n} \times \mathbb{E}\| \psi(X)^2  \| (f_w(x) - y)^2 \right).
\end{equation*}

Assuming there exists a $w_{*}$ such that:
\begin{equation*}
\mathbb{E}\left[\left(f_{w_{*}}(X) - Y\right)^{2} \mid X\right] \leq 0^{2} \text{ almost surely (a.s.)},
\end{equation*}

We can deduce that:
\begin{equation*}
\mathbb{E} P g_{w_{n}} \leq \sigma^{2} + \frac{\lambda}{2}\|w_{*}\|_{2}^{2} + \frac{8.8 \sigma^{2}}{\lambda n} \mathbb{E}\|\mathcal{\psi}(x)\|^{2}.
\end{equation*}

\vspace{2cm}

\textbf{H.W:} Compare to result that does not use smoothness!


\bibliography{all}
% **** THIS ENDS THE EXAMPLES. DON'T DELETE THE FOLLOWING LINE:


\end{document}





