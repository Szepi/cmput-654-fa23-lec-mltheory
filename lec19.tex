\documentclass[twoside]{article}

% We add packages, macros here:
%!TEX root =  lec-template.tex
\usepackage{lmodern}
\usepackage[english]{babel}
\usepackage{latexsym}
\usepackage{amsmath}
\usepackage{mathrsfs}
\usepackage{amssymb}
\usepackage{mathtools}
\usepackage[inline,shortlabels]{enumitem} % we prefer enumitem because of its margin adjustment caps
\usepackage{bm}
\usepackage{datetime}
\usepackage[table,xcdraw]{xcolor}
\usepackage{accents}
\usepackage{tikz}
\usepackage{listings}
\usepackage{mdframed}
\usepackage{pgfplots}
\usepackage{pgfplotstable}
\usepackage[boxed]{algorithm}
\usepackage{algpseudocode}
\usepackage{dsfont}
\usepackage{color}
\usepackage{colortbl}
\usepackage{pifont}
\usepackage[bf,font=small,singlelinecheck=off]{caption}
\usepackage{microtype} % improved spacing between words for easier reading
\usepackage{float}
\usepackage{xfrac} % sfrac
\usepackage{xspace}

\linespread{1.1}


\usepackage[textsize=tiny]{todonotes}
\makeatletter
\renewcommand{\todo}[2][]{\@todo[#1]{#2}}
\makeatother

\setlength{\marginparwidth}{10ex}
\newcommand{\todoc}[2][]{\todo[size=\scriptsize,color=blue!20!white,#1]{Csaba: #2}}
\newcommand{\todoj}[2][]{\todo[size=\scriptsize,color=red!20!white,#1]{Jincheng: #2}}

\usepackage{hyperref}
\hypersetup{
    unicode=false,          % non-Latin characters in Acrobat�s bookmarks
    pdftoolbar=true,        % show Acrobat�s toolbar?
    pdfmenubar=true,        % show Acrobat�s menu?
    pdffitwindow=false,     % window fit to page when opened
    pdfstartview={FitH},    % fits the width of the page to the window
    pdftitle={},    % title
    pdfauthor={},     % author
    pdfsubject={Theory, Machine Learning, Lectures},   % subject of the document
    pdfcreator={},   % creator of the document
    pdfproducer={}, % producer of the document
    pdfkeywords={theory} {machine learning} {lecture notes} {CMPUT 654} {Fall 2023}, % list of keywords
    pdfnewwindow=true,      % links in new window
    colorlinks=true,       % false: boxed links; true: colored links
    linkcolor=blue,          % color of internal links (change box color with linkbordercolor)
    citecolor=blue,        % color of links to bibliography
    filecolor=magenta,      % color of file links
    urlcolor=cyan           % color of external links
}
\usepackage{amsthm}
\usepackage{times}
\usepackage{natbib}
\usepackage{nicefrac}
\usepackage{wrapfig}
\usepackage[capitalize]{cleveref}
\usepackage[nottoc,numbib]{tocbibind}

%\usepackage[bmargin=0.75in]{geometry}
\usepackage[margin=1.1in]{geometry}
\usepackage[normalem]{ulem}


%%%%%%%%%%%%%%%%%%%%%%%%%%%%%%%%
% HYPHENATION
%%%%%%%%%%%%%%%%%%%%%%%%%%%%%%%%

\pretolerance=5000
\tolerance=9000
\emergencystretch=0pt
\righthyphenmin=4
\lefthyphenmin=4


\bibliographystyle{plainnat}

\newcommand{\E}{\mathbb{E}}
\newcommand{\EE}[1]{\E[#1]}
\newcommand{\PP}{\mathbb{P}}
\newcommand{\Prob}[1]{\mathbb{P}(#1)}
\newcommand{\one}[1]{\mathbb{I}\{#1\}}
\newcommand{\Supp}{\operatorname{supp}}
\newcommand{\ip}[1]{\langle #1 \rangle}
\newcommand{\bip}[1]{\left\langle #1 \right\rangle}
\newcommand{\norm}[1]{\|#1\|}
\newcommand{\R}{\mathbb{R}}
\newcommand{\N}{\mathbb{N}}
\newcommand{\cA}{\mathcal{A}}
\newcommand{\cB}{\mathcal{B}}
\newcommand{\cC}{\mathcal{C}}
\newcommand{\cD}{\mathcal{D}}
\newcommand{\cE}{\mathcal{E}}
\newcommand{\cF}{\mathcal{F}}
\newcommand{\cG}{\mathcal{G}}
\newcommand{\cH}{\mathcal{H}}
\newcommand{\cK}{\mathcal{K}}
\newcommand{\cL}{\mathcal{L}}
\newcommand{\cN}{\mathcal{N}}
\newcommand{\cP}{\mathcal{P}}
\newcommand{\cQ}{\mathcal{Q}}
\newcommand{\cR}{\mathcal{R}}
\newcommand{\cS}{\mathcal{S}}
\newcommand{\sA}{\mathscr A}

\newcommand{\cM}{\mathcal{M}}
\newcommand{\cX}{\mathcal{X}}
\newcommand{\cY}{\mathcal{Y}}
\newcommand{\NN}{\mathbb{N}}
\newcommand{\RR}{\mathbb{R}}
\newcommand{\VV}[1]{\mathbb{V}[#1]}

\newcommand{\epsapp}{\epsilon}
\newcommand{\epssub}{\delta}

\DeclareMathOperator{\Range}{range}
\newcommand{\rows}{\operatorname{rows}}

\renewcommand{\epsilon}{\varepsilon}
\newcommand{\ceil}[1]{\left\lceil {#1} \right\rceil}
\newcommand{\floor}[1]{\left\lfloor {#1} \right\rfloor}
\newcommand{\ones}{\mathbf{1}}
\newcommand{\zeros}{\mathbf{0}}
\DeclareMathOperator*{\argmin}{arg\ min}
\DeclareMathOperator*{\argmax}{arg\ max}


\def\rvzero{{\mathbf{0}}}
\def\rvone{{\mathbf{1}}}

\def\identiymatrix{\mathbf{Id}}

\newcommand{\softmax}{\mathrm{softmax}}
\newcommand{\KL}{D_{\mathrm{KL}}}

% Graph
\def\gA{{\mathcal{A}}}
\def\gB{{\mathcal{B}}}
\def\gC{{\mathcal{C}}}
\def\gD{{\mathcal{D}}}
\def\gE{{\mathcal{E}}}
\def\gF{{\mathcal{F}}}
\def\gG{{\mathcal{G}}}
\def\gH{{\mathcal{H}}}
\def\gI{{\mathcal{I}}}
\def\gJ{{\mathcal{J}}}
\def\gK{{\mathcal{K}}}
\def\gL{{\mathcal{L}}}
\def\gM{{\mathcal{M}}}
\def\gN{{\mathcal{N}}}
\def\gO{{\mathcal{O}}}
\def\gP{{\mathcal{P}}}
\def\gQ{{\mathcal{Q}}}
\def\gR{{\mathcal{R}}}
\def\gS{{\mathcal{S}}}
\def\gT{{\mathcal{T}}}
\def\gU{{\mathcal{U}}}
\def\gV{{\mathcal{V}}}
\def\gW{{\mathcal{W}}}
\def\gX{{\mathcal{X}}}
\def\gY{{\mathcal{Y}}}
\def\gZ{{\mathcal{Z}}}

% Sets
\def\sA{{\mathbb{A}}}
\def\sB{{\mathbb{B}}}
\def\sC{{\mathbb{C}}}
\def\sD{{\mathbb{D}}}
% Don't use a set called E, because this would be the same as our symbol
% for expectation.
\def\sE{{\mathbb{E}}}
\def\sF{{\mathbb{F}}}
\def\sG{{\mathbb{G}}}
\def\sH{{\mathbb{H}}}
\def\sI{{\mathbb{I}}}
\def\sJ{{\mathbb{J}}}
\def\sK{{\mathbb{K}}}
\def\sL{{\mathbb{L}}}
\def\sM{{\mathbb{M}}}
\def\sN{{\mathbb{N}}}
\def\sO{{\mathbb{O}}}
\def\sP{{\mathbb{P}}}
\def\sQ{{\mathbb{Q}}}
\def\sR{{\mathbb{R}}}
\def\sS{{\mathbb{S}}}
\def\sT{{\mathbb{T}}}
\def\sU{{\mathbb{U}}}
\def\sV{{\mathbb{V}}}
\def\sW{{\mathbb{W}}}
\def\sX{{\mathbb{X}}}
\def\sY{{\mathbb{Y}}}
\def\sZ{{\mathbb{Z}}}

\newcommand{\dimE}{\mathrm{dim}_{\mathcal{E}}}
\DeclareMathOperator{\diam}{diam}


%%
%% ADD PACKAGES here:
%%
%
%\usepackage{amsmath,amsfonts,graphicx}
%
%
% The following commands set up the lecnum (lecture number)
% counter and make various numbering schemes work relative
% to the lecture number.
%

\newcounter{lecnum}
\renewcommand{\thepage}{\thelecnum-\arabic{page}}
\renewcommand{\thesection}{\thelecnum.\arabic{section}}
\renewcommand{\theequation}{\thelecnum.\arabic{equation}}
\renewcommand{\thefigure}{\thelecnum.\arabic{figure}}
\renewcommand{\thetable}{\thelecnum.\arabic{table}}


%
% The following macro is used to generate the header.
%
\newcommand{\lecture}[4]{
   \pagestyle{myheadings}
   \thispagestyle{plain}
   \newpage
   \setcounter{lecnum}{#1}
   \setcounter{page}{1}
   \noindent
   \begin{center}
   \framebox{
      \vbox{\vspace{2mm}
    \hbox to 6.28in { {\bf CMPUT 654 Fa 23: Theoretical Foundations of Machine Learning \hfill Fall 2023} }
       \vspace{4mm}
       \hbox to 6.28in { {\Large \hfill Lecture #1: #2  \hfill} }
       \vspace{2mm}
       \hbox to 6.28in { {\it Lecturer: #3 \hfill Scribes: #4} }
      \vspace{2mm}}
   }
   \end{center}
   \markboth{Lecture #1: #2}{Lecture #1: #2}

   \noindent {\bf Note}: {\it 
   \LaTeX\  template courtesy of UC Berkeley EECS dept. (\href{https://inst.eecs.berkeley.edu/~cs294-8/sp03/Materials/}{link} to directory)
   }

   \noindent {\bf Disclaimer}: {\it These notes have \underline{\textbf{not}} been subjected to the
   usual scrutiny reserved for formal publications. They may be
   distributed outside this class only with the permission of the
   Instructor.} \vspace*{4mm}
}
%
% Convention for citations is authors' initials followed by the year.
% For example, to cite a paper by Leighton and Maggs you would type
% \cite{LM89}, and to cite a paper by Strassen you would type \cite{S69}.
%%%%%%%%% (To avoid bibliography problems, for now we redefine the \cite command.)
%%%%%%%%% Also commands that create a suitable format for the reference list.
%%%%%%%%\renewcommand{\cite}[1]{[#1]}
%%%%%%%%\def\beginrefs{\begin{list}%
%%%%%%%%        {[\arabic{equation}]}{\usecounter{equation}
%%%%%%%%         \setlength{\leftmargin}{2.0truecm}\setlength{\labelsep}{0.4truecm}%
%%%%%%%%         \setlength{\labelwidth}{1.6truecm}}}
%%%%%%%%\def\endrefs{\end{list}}
%%%%%%%%\def\bibentry#1{\item[\hbox{[#1]}]}

%Use this command for a figure; it puts a figure in wherever you want it.
%usage: \fig{NUMBER}{SPACE-IN-INCHES}{CAPTION}
\newcommand{\fig}[3]{
			\vspace{#2}
			\begin{center}
			Figure \thelecnum.#1:~#3
			\end{center}
	}
% Use these for theorems, lemmas, proofs, etc.

%!TEX root =  lec-template.tex
%%%%%%%%%%%%%%%%%%%%%%%%%%%%%%%%
% THEOREMS
%%%%%%%%%%%%%%%%%%%%%%%%%%%%%%%%
\theoremstyle{plain}
\newtheorem{theorem}{Theorem}[lecnum]
\newtheorem{claim}[theorem]{Claim}
\newtheorem{proposition}[theorem]{Proposition}
\newtheorem{lemma}[theorem]{Lemma}
\newtheorem{corollary}[theorem]{Corollary}
\newtheorem{example}[theorem]{Example}
\theoremstyle{definition}
\newtheorem{definition}[theorem]{Definition}
\newtheorem{assumption}[theorem]{Assumption}
\newtheorem{remark}[theorem]{Remark}
\newtheorem{exercise}[theorem]{Exercise}
\theoremstyle{remark}


%\newtheorem{theorem}{Theorem}[lecnum]
%\newtheorem{lemma}[theorem]{Lemma}
%\newtheorem{proposition}[theorem]{Proposition}
%\newtheorem{claim}[theorem]{Claim}
%\newtheorem{corollary}[theorem]{Corollary}
%\newtheorem{definition}[theorem]{Definition}
%\newenvironment{proof}{{\bf Proof:}}{\hfill\rule{2mm}{2mm}}

% **** IF YOU WANT TO DEFINE ADDITIONAL MACROS FOR YOURSELF, PUT THEM HERE:

\newcommand\myE{\mathbb{E}}

\begin{document}
%FILL IN THE RIGHT INFO.
%\lecture{**LECTURE-NUMBER**}{**DATE**}{**LECTURER**}{**SCRIBE**}
\lecture{19}{November 9}{Csaba Szepesv\'ari}{Aniket Sharma}
%\footnotetext{These notes are partially based on those of Nigel Mansell.}

% **** YOUR NOTES GO HERE:

% Some general latex examples and examples making use of the
% macros follow.  
%**** IN GENERAL, BE BRIEF. LONG SCRIBE NOTES, NO MATTER HOW WELL WRITTEN,
%**** ARE NEVER READ BY ANYBODY.
% This lecture's notes illustrate some uses of
% various \LaTeX\ macros.  
% Take a look at this and imitate.

\begin{center}
    \href{https://youtu.be/A_Ut1Q-RN2U?feature=shared}{Lecture 19 video}    
\end{center}

\section{Outline}
\begin{itemize}
    \item Model Selection Problem
    \item Model Selection using Validation Data
    \item Model Selection using Training Data
    \item Bayesian Model Selection and Averaging
\end{itemize}

\section{Model Selection Problem}
We have a set of function classes $\cG_i \in \R^z, i \in \N$, and
\begin{align*}
    g_n^{(i)} &= argmin_{g \in \cG_i} P_ng \\
    Pg_n^{(i)} &\leq inf_{g \in \cG_i} Pg + penalty_i(n, \delta) \text{\hspace{1cm}, wp } 1-\delta\\
    Pg_n &= min_i Pg_n^{(i)}
\end{align*}

We want to find the class such that the empirical performance is the best
\begin{align*}
    g_n \in argmin_{g \in \cup_i \cG_i} Png
\end{align*}

\textit{Note:} If $VC(\cG_i) = d_i$, then $penalty_i(n) = \sqrt{\frac{d_i ln\left(\frac{1}{\delta}\right)}{n}}$.

\section{Model Selection using Validation Data}
We have $z_{1:n}, z'_{1:m} \sim P^{\otimes(n+m)}$, where $z_{1:n}$ is the training data and $z'_{1:m}$ is the validation data.
\begin{align*}
    P'_m &= \frac{1}{m} \Sigma_{i=1}^{m} \delta_{z'_i} \\
    I &= argmin_{i \in \N} P'_m g_n^{(i)} + \sqrt{ln\left(\frac{1}{q_i}\right)}
\end{align*}

Here, $\sqrt{ln\left(\frac{1}{q_i}\right)}$ is the ``complexity" penalty. Also, $\Sigma q_i \leq 1, q_i \geq 0$. A typical choice will be $q_i = \frac{1}{i(i+1)}$ or $q_i = \frac{1}{(i+1)^2}$.

We want to consider less complex classes first (Occam's razor) like $d_1 \leq d_2 \leq \dots$ for VC classes.

\begin{theorem}
    Let $sup_{z, z'} sup_{g \in \cup_i \cG_i} g(z) - g(z') \leq M$, then
    \begin{enumerate}
        \item wp $1-\delta$,
            \begin{align*}
                Pg_n^{I} \leq inf_{i \in \N} P'_m g_n^{i} + \sqrt{ln\left(\frac{1}{q_i}\right)} + M\sqrt{\frac{ln\left(\frac{1}{\delta}\right)}{2m}}
            \end{align*}
        \item wp $1-\delta$,
            \begin{align*}
                Pg_n^{I} \leq inf_{i \in \N} Pg_n^{i} + \sqrt{ln\left(\frac{1}{q_i}\right)} + M\sqrt{\frac{ln\left(\frac{2}{\delta}\right)}{2m}}
            \end{align*}
    \end{enumerate}
\end{theorem}

\section{Model Selection using Training Data}
An alternative approach would be to use the training data for model selection instead of splitting.

\begin{align*}
    (I, \cG) := argmin \{P_n g + R_i(g, z_{1:n}) : i \in \N, g \in \cG_i\}
\end{align*}

Here, $R_i(g, z_{1:n})$ is the data-dependent penalty.

\begin{theorem}
    $\Sigma q_i \leq 1, q_i \geq 0, \forall \delta \in (0, 1)$,
    \begin{align*}
        \alpha P_g \leq P_n g + \epsilon_i(g, z_{1:n}) + \left(\frac{ln\left(\frac{c_0}{\delta}\right)}{\lambda n}\right)^{\beta}
    \end{align*}

    for some $\alpha, \beta, \lambda > 0, c_0 \geq 1$,
    \begin{align*}
        R_i(g, z_{1:n}) \geq \epsilon_i(g, z_{1:n}) + 2^{max(0, \beta - 1)}\left(\frac{ln\left(\frac{c_0}{q_i}\right)}{\lambda n}\right)^{\beta}
    \end{align*}

    Part 1: $\forall \delta \in (0, 1)$ wp $1-\delta$: $\forall i \in \N, g \in \cG$,
    \begin{align*}
        \alpha P_g \leq P_n g + R_i(g, z_{1:n}) + 2^{max(0, \beta - 1)}\left(\frac{ln\left(\frac{c_0}{q_i}\right)}{\lambda n}\right)^{\beta}
    \end{align*}

    Part 2: $\forall \delta \in (0, 1), \forall i \in \N, g \in \cG$,
    \begin{align*}
        P_n g + R_i(g, z_{1:n}) \leq \EE{\alpha'P_n g + \alpha''R_i(g, z_{1:n})} + \epsilon'_i(g, \delta)
    \end{align*}
    then wp $1-\delta$,
    \begin{align*}
        \alpha PG \leq inf_{i \in \N, g \in \cG_i} \left[\alpha' Pg + \alpha'' \EE{R_i(g, z_{1:n})} + \epsilon'\left(g, \frac{\delta}{2}\right)\right] + 2^{max(0, \beta - 1)}\left(\frac{ln\left(\frac{c_0}{q_i}\right)}{\lambda n}\right)^{\beta}
    \end{align*}
\end{theorem}

\subsection{Concentration of Empirical Rademacher Complexity}
\begin{theorem}
    \begin{align*}
        R_i(g, z_{1:n}) \geq 2R_n(\cG_i, P) + M_i\sqrt{\frac{ln\left(\frac{1}{q_i}\right)}{2n}}
    \end{align*}
    where, $M_i = sup_{g \in \cG_i} sup_{z, z' \in \cZ} g(z) - g(z')$. Then,
    \begin{enumerate}
        \item wp $1-\delta$: $i \in \N, g \in \cG_i$,
        \begin{align*}
            Pg \leq P_n g + R_i(g, z_{1:n}) + M_i\sqrt{\frac{ln\left(\frac{1}{\delta}\right)}{2n}}
        \end{align*}
        \item wp $1-\delta$,
        \begin{align*}
            PG \leq inf_{i \in \N, g \in \cG_i} Pg + R_i(g, z_{1:n}) + 2M_i\sqrt{\frac{ln\left(\frac{2}{\delta}\right)}{2n}}
        \end{align*}
    \end{enumerate}
\end{theorem}

\begin{theorem}
    $M \geq sup_{g} sup_{z, z'} g(z) - g(z')$, then wp $1-\delta$,
    \begin{align*}
        R_n(\cG, P) \leq R(\cG, z_{1:n}) + M\sqrt{\frac{ln\left(\frac{1}{\delta}\right)}{2n}}
    \end{align*}
    Also wp $1-\delta$,
    \begin{align*}
        R_n(\cG, P) \geq R(\cG, z_{1:n}) - M\sqrt{\frac{ln\left(\frac{1}{\delta}\right)}{2n}}
    \end{align*}
\end{theorem}

Here, $R(\cG, z_{1:n})$ is the empirical Rademacher complexity.

\begin{corollary}
    wp $1-\delta$: $\forall g \in \cG$,
    \begin{align*}
        Pg \leq P_n g + 2R(\cG, z_{1:n}) + 3M\sqrt{\frac{ln\left(\frac{2}{\delta}\right)}{2n}}
    \end{align*}
\end{corollary}

\begin{theorem}
    \begin{align*}
        R_i(z_{1:n}) \geq R(\cG_i, z_{1:n} + 3M_i\sqrt{\frac{ln\left(\frac{2}{q_i}\right)}{2n}}
    \end{align*}
    Then,
    \begin{enumerate}
        \item wp $1-\delta$: $\forall i \in \N, g \in \cG_i$,
        \begin{align*}
            Pg \leq P_n g + R_i(z_{1:n}) + 3M_i\sqrt{\frac{ln\left(\frac{1}{\delta}\right)}{2n}}
        \end{align*}
        \item wp $1-\delta$,
        \begin{align*}
            PG \leq inf_{i \in \N, g \in \cG_i} Pg + \EE{R_i(z_{1:n})} + 4M_i\sqrt{\frac{ln\left(\frac{2}{\delta}\right)}{2n}}
        \end{align*}
    \end{enumerate}
\end{theorem}

\section{Bayesian Model Selection and Averaging}
Consider the Gibb's algorithm,
\begin{align*}
    g \sim exp(-\beta n P_n g) \pi_0(dg)
\end{align*}

Here, $g \in \cG$ and $\pi_0(dg)$ is the prior.

Take $\Sigma q_i = 1$,
\begin{align*}
    (I, \cG) \sim P_i \pi_i(dg) exp(-\beta n P_n g)
\end{align*}

Here, $\pi_i(dg)$ is the prior for class $\cG_i$.

Now, we can use the Bayesian formula for Gibbs model selection and select a model randomly but in practice model averaging often leads to superior performance.

For $f \in \cF \subseteq \R^\cX$,
\begin{align*}
    \tilde{P_n}(df, i) = P_i \pi_i(dg) exp(-\beta n P_n l(f))
\end{align*}

Here, $\tilde{P_n}(df, i)$ is the posterior.

Then we can make the predictions using,
\begin{align*}
    \Sigma_i \int f(x) \tilde{P_n}(df, i)
\end{align*}

\textit{Claim:} Averaging $>>>$ Any Model Selection

% **** THIS ENDS THE EXAMPLES. DON'T DELETE THE FOLLOWING LINE:

\end{document}
