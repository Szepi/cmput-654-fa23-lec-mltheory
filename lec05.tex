\documentclass[twoside]{article}

% We add packages, macros here:
%!TEX root =  lec-template.tex
\usepackage{lmodern}
\usepackage[english]{babel}
\usepackage{latexsym}
\usepackage{amsmath}
\usepackage{mathrsfs}
\usepackage{amssymb}
\usepackage{mathtools}
\usepackage[inline,shortlabels]{enumitem} % we prefer enumitem because of its margin adjustment caps
\usepackage{bm}
\usepackage{datetime}
\usepackage[table,xcdraw]{xcolor}
\usepackage{accents}
\usepackage{tikz}
\usepackage{listings}
\usepackage{mdframed}
\usepackage{pgfplots}
\usepackage{pgfplotstable}
\usepackage[boxed]{algorithm}
\usepackage{algpseudocode}
\usepackage{dsfont}
\usepackage{color}
\usepackage{colortbl}
\usepackage{pifont}
\usepackage[bf,font=small,singlelinecheck=off]{caption}
\usepackage{microtype} % improved spacing between words for easier reading
\usepackage{float}
\usepackage{xfrac} % sfrac
\usepackage{xspace}

\linespread{1.1}


\usepackage[textsize=tiny]{todonotes}
\makeatletter
\renewcommand{\todo}[2][]{\@todo[#1]{#2}}
\makeatother

\setlength{\marginparwidth}{10ex}
\newcommand{\todoc}[2][]{\todo[size=\scriptsize,color=blue!20!white,#1]{Csaba: #2}}
\newcommand{\todoj}[2][]{\todo[size=\scriptsize,color=red!20!white,#1]{Jincheng: #2}}

\usepackage{hyperref}
\hypersetup{
    unicode=false,          % non-Latin characters in Acrobat�s bookmarks
    pdftoolbar=true,        % show Acrobat�s toolbar?
    pdfmenubar=true,        % show Acrobat�s menu?
    pdffitwindow=false,     % window fit to page when opened
    pdfstartview={FitH},    % fits the width of the page to the window
    pdftitle={},    % title
    pdfauthor={},     % author
    pdfsubject={Theory, Machine Learning, Lectures},   % subject of the document
    pdfcreator={},   % creator of the document
    pdfproducer={}, % producer of the document
    pdfkeywords={theory} {machine learning} {lecture notes} {CMPUT 654} {Fall 2023}, % list of keywords
    pdfnewwindow=true,      % links in new window
    colorlinks=true,       % false: boxed links; true: colored links
    linkcolor=blue,          % color of internal links (change box color with linkbordercolor)
    citecolor=blue,        % color of links to bibliography
    filecolor=magenta,      % color of file links
    urlcolor=cyan           % color of external links
}
\usepackage{amsthm}
\usepackage{times}
\usepackage{natbib}
\usepackage{nicefrac}
\usepackage{wrapfig}
\usepackage[capitalize]{cleveref}
\usepackage[nottoc,numbib]{tocbibind}

%\usepackage[bmargin=0.75in]{geometry}
\usepackage[margin=1.1in]{geometry}
\usepackage[normalem]{ulem}


%%%%%%%%%%%%%%%%%%%%%%%%%%%%%%%%
% HYPHENATION
%%%%%%%%%%%%%%%%%%%%%%%%%%%%%%%%

\pretolerance=5000
\tolerance=9000
\emergencystretch=0pt
\righthyphenmin=4
\lefthyphenmin=4


\bibliographystyle{plainnat}

\newcommand{\E}{\mathbb{E}}
\newcommand{\EE}[1]{\E[#1]}
\newcommand{\PP}{\mathbb{P}}
\newcommand{\Prob}[1]{\mathbb{P}(#1)}
\newcommand{\one}[1]{\mathbb{I}\{#1\}}
\newcommand{\Supp}{\operatorname{supp}}
\newcommand{\ip}[1]{\langle #1 \rangle}
\newcommand{\bip}[1]{\left\langle #1 \right\rangle}
\newcommand{\norm}[1]{\|#1\|}
\newcommand{\R}{\mathbb{R}}
\newcommand{\N}{\mathbb{N}}
\newcommand{\cA}{\mathcal{A}}
\newcommand{\cB}{\mathcal{B}}
\newcommand{\cC}{\mathcal{C}}
\newcommand{\cD}{\mathcal{D}}
\newcommand{\cE}{\mathcal{E}}
\newcommand{\cF}{\mathcal{F}}
\newcommand{\cG}{\mathcal{G}}
\newcommand{\cH}{\mathcal{H}}
\newcommand{\cK}{\mathcal{K}}
\newcommand{\cL}{\mathcal{L}}
\newcommand{\cN}{\mathcal{N}}
\newcommand{\cP}{\mathcal{P}}
\newcommand{\cQ}{\mathcal{Q}}
\newcommand{\cR}{\mathcal{R}}
\newcommand{\cS}{\mathcal{S}}
\newcommand{\sA}{\mathscr A}

\newcommand{\cM}{\mathcal{M}}
\newcommand{\cX}{\mathcal{X}}
\newcommand{\cY}{\mathcal{Y}}
\newcommand{\NN}{\mathbb{N}}
\newcommand{\RR}{\mathbb{R}}
\newcommand{\VV}[1]{\mathbb{V}[#1]}

\newcommand{\epsapp}{\epsilon}
\newcommand{\epssub}{\delta}

\DeclareMathOperator{\Range}{range}
\newcommand{\rows}{\operatorname{rows}}

\renewcommand{\epsilon}{\varepsilon}
\newcommand{\ceil}[1]{\left\lceil {#1} \right\rceil}
\newcommand{\floor}[1]{\left\lfloor {#1} \right\rfloor}
\newcommand{\ones}{\mathbf{1}}
\newcommand{\zeros}{\mathbf{0}}
\DeclareMathOperator*{\argmin}{arg\ min}
\DeclareMathOperator*{\argmax}{arg\ max}


\def\rvzero{{\mathbf{0}}}
\def\rvone{{\mathbf{1}}}

\def\identiymatrix{\mathbf{Id}}

\newcommand{\softmax}{\mathrm{softmax}}
\newcommand{\KL}{D_{\mathrm{KL}}}

% Graph
\def\gA{{\mathcal{A}}}
\def\gB{{\mathcal{B}}}
\def\gC{{\mathcal{C}}}
\def\gD{{\mathcal{D}}}
\def\gE{{\mathcal{E}}}
\def\gF{{\mathcal{F}}}
\def\gG{{\mathcal{G}}}
\def\gH{{\mathcal{H}}}
\def\gI{{\mathcal{I}}}
\def\gJ{{\mathcal{J}}}
\def\gK{{\mathcal{K}}}
\def\gL{{\mathcal{L}}}
\def\gM{{\mathcal{M}}}
\def\gN{{\mathcal{N}}}
\def\gO{{\mathcal{O}}}
\def\gP{{\mathcal{P}}}
\def\gQ{{\mathcal{Q}}}
\def\gR{{\mathcal{R}}}
\def\gS{{\mathcal{S}}}
\def\gT{{\mathcal{T}}}
\def\gU{{\mathcal{U}}}
\def\gV{{\mathcal{V}}}
\def\gW{{\mathcal{W}}}
\def\gX{{\mathcal{X}}}
\def\gY{{\mathcal{Y}}}
\def\gZ{{\mathcal{Z}}}

% Sets
\def\sA{{\mathbb{A}}}
\def\sB{{\mathbb{B}}}
\def\sC{{\mathbb{C}}}
\def\sD{{\mathbb{D}}}
% Don't use a set called E, because this would be the same as our symbol
% for expectation.
\def\sE{{\mathbb{E}}}
\def\sF{{\mathbb{F}}}
\def\sG{{\mathbb{G}}}
\def\sH{{\mathbb{H}}}
\def\sI{{\mathbb{I}}}
\def\sJ{{\mathbb{J}}}
\def\sK{{\mathbb{K}}}
\def\sL{{\mathbb{L}}}
\def\sM{{\mathbb{M}}}
\def\sN{{\mathbb{N}}}
\def\sO{{\mathbb{O}}}
\def\sP{{\mathbb{P}}}
\def\sQ{{\mathbb{Q}}}
\def\sR{{\mathbb{R}}}
\def\sS{{\mathbb{S}}}
\def\sT{{\mathbb{T}}}
\def\sU{{\mathbb{U}}}
\def\sV{{\mathbb{V}}}
\def\sW{{\mathbb{W}}}
\def\sX{{\mathbb{X}}}
\def\sY{{\mathbb{Y}}}
\def\sZ{{\mathbb{Z}}}

\newcommand{\dimE}{\mathrm{dim}_{\mathcal{E}}}
\DeclareMathOperator{\diam}{diam}


%%
%% ADD PACKAGES here:
%%
%
%\usepackage{amsmath,amsfonts,graphicx}
%
%
% The following commands set up the lecnum (lecture number)
% counter and make various numbering schemes work relative
% to the lecture number.
%

\newcounter{lecnum}
\renewcommand{\thepage}{\thelecnum-\arabic{page}}
\renewcommand{\thesection}{\thelecnum.\arabic{section}}
\renewcommand{\theequation}{\thelecnum.\arabic{equation}}
\renewcommand{\thefigure}{\thelecnum.\arabic{figure}}
\renewcommand{\thetable}{\thelecnum.\arabic{table}}


%
% The following macro is used to generate the header.
%
\newcommand{\lecture}[4]{
   \pagestyle{myheadings}
   \thispagestyle{plain}
   \newpage
   \setcounter{lecnum}{#1}
   \setcounter{page}{1}
   \noindent
   \begin{center}
   \framebox{
      \vbox{\vspace{2mm}
    \hbox to 6.28in { {\bf CMPUT 654 Fa 23: Theoretical Foundations of Machine Learning \hfill Fall 2023} }
       \vspace{4mm}
       \hbox to 6.28in { {\Large \hfill Lecture #1: #2  \hfill} }
       \vspace{2mm}
       \hbox to 6.28in { {\it Lecturer: #3 \hfill Scribes: #4} }
      \vspace{2mm}}
   }
   \end{center}
   \markboth{Lecture #1: #2}{Lecture #1: #2}

   \noindent {\bf Note}: {\it 
   \LaTeX\  template courtesy of UC Berkeley EECS dept. (\href{https://inst.eecs.berkeley.edu/~cs294-8/sp03/Materials/}{link} to directory)
   }

   \noindent {\bf Disclaimer}: {\it These notes have \underline{\textbf{not}} been subjected to the
   usual scrutiny reserved for formal publications. They may be
   distributed outside this class only with the permission of the
   Instructor.} \vspace*{4mm}
}
%
% Convention for citations is authors' initials followed by the year.
% For example, to cite a paper by Leighton and Maggs you would type
% \cite{LM89}, and to cite a paper by Strassen you would type \cite{S69}.
%%%%%%%%% (To avoid bibliography problems, for now we redefine the \cite command.)
%%%%%%%%% Also commands that create a suitable format for the reference list.
%%%%%%%%\renewcommand{\cite}[1]{[#1]}
%%%%%%%%\def\beginrefs{\begin{list}%
%%%%%%%%        {[\arabic{equation}]}{\usecounter{equation}
%%%%%%%%         \setlength{\leftmargin}{2.0truecm}\setlength{\labelsep}{0.4truecm}%
%%%%%%%%         \setlength{\labelwidth}{1.6truecm}}}
%%%%%%%%\def\endrefs{\end{list}}
%%%%%%%%\def\bibentry#1{\item[\hbox{[#1]}]}

%Use this command for a figure; it puts a figure in wherever you want it.
%usage: \fig{NUMBER}{SPACE-IN-INCHES}{CAPTION}
\newcommand{\fig}[3]{
			\vspace{#2}
			\begin{center}
			Figure \thelecnum.#1:~#3
			\end{center}
	}
% Use these for theorems, lemmas, proofs, etc.

%!TEX root =  lec-template.tex
%%%%%%%%%%%%%%%%%%%%%%%%%%%%%%%%
% THEOREMS
%%%%%%%%%%%%%%%%%%%%%%%%%%%%%%%%
\theoremstyle{plain}
\newtheorem{theorem}{Theorem}[lecnum]
\newtheorem{claim}[theorem]{Claim}
\newtheorem{proposition}[theorem]{Proposition}
\newtheorem{lemma}[theorem]{Lemma}
\newtheorem{corollary}[theorem]{Corollary}
\newtheorem{example}[theorem]{Example}
\theoremstyle{definition}
\newtheorem{definition}[theorem]{Definition}
\newtheorem{assumption}[theorem]{Assumption}
\newtheorem{remark}[theorem]{Remark}
\newtheorem{exercise}[theorem]{Exercise}
\theoremstyle{remark}


%\newtheorem{theorem}{Theorem}[lecnum]
%\newtheorem{lemma}[theorem]{Lemma}
%\newtheorem{proposition}[theorem]{Proposition}
%\newtheorem{claim}[theorem]{Claim}
%\newtheorem{corollary}[theorem]{Corollary}
%\newtheorem{definition}[theorem]{Definition}
%\newenvironment{proof}{{\bf Proof:}}{\hfill\rule{2mm}{2mm}}

% **** IF YOU WANT TO DEFINE ADDITIONAL MACROS FOR YOURSELF, PUT THEM HERE:

\newcommand\myE{\mathbb{E}}

\begin{document}
%FILL IN THE RIGHT INFO.
%\lecture{**LECTURE-NUMBER**}{**DATE**}{**LECTURER**}{**SCRIBE**}
\lecture{5}{ERMs and Learning the AND Class (Sept 19)}{Csaba Szepesv\'ari}{Zixin Zhong}
%\footnotetext{These notes are partially based on those of Nigel Mansell.}

% **** YOUR NOTES GO HERE:

% Some general latex examples and examples making use of the
% macros follow.  
%**** IN GENERAL, BE BRIEF. LONG SCRIBE NOTES, NO MATTER HOW WELL WRITTEN,
%**** ARE NEVER READ BY ANYBODY.


\section{Review}

\subsection{Recall: Concentration inequalities}

\begin{theorem}[Additive Chernoff's Inequality]
    Let $X_1,\ldots, X_n\in [0,1]$ be i.i.d.\ random variables, $\bar{X}_n=\frac{1}{n}(X_1+\ldots+X_n)$, $\mu = \mathbb{E}X_1$. We have
    \begin{enumerate}[(a)]
        \item  $\forall \delta\in (0,1)$, with probability $1-\delta$,
    \begin{align*}
        \bar{X}_n \le \mu + \sqrt{\frac{\log(1/\delta)}{2n}};
    \end{align*}
        \item $\forall \delta\in (0,1)$, with probability $1-\delta$,
    \begin{align*}
        \bar{X}_n \ge \mu - \sqrt{\frac{\log(1/\delta)}{2n}}.
    \end{align*}
    \end{enumerate}
   
\end{theorem}

\begin{theorem}[Multiplicative Chernoff's Inequality]
    Let $X_1,\ldots, X_n\in [0,1]$ be i.i.d.\ random variables, $\bar{X}_n=\frac{1}{n}(X_1+\ldots+X_n)$, $\mu = \mathbb{E}X_1$. We have
    \begin{enumerate}[(a)]
        \item $\forall \delta\in (0,1)$, with probability $1-\delta$,
    \begin{align*}
        \bar{X}_n \le \mu + \sqrt{\frac{2\mu\log(1/\delta)}{n}}+\frac{1}{3n};  
    \end{align*}
    \item $\forall \delta\in (0,1)$, with probability $1-\delta$,
    \begin{align*}
        \bar{X}_n \ge \mu - \sqrt{\frac{2\mu\log(1/\delta)}{n}}.    \qquad\qquad \text{(*)}
    \end{align*}
    \end{enumerate}
  
\end{theorem}

\subsection{Recall: PAC-learning}

Let function  $f_*: \{0,1\}^d \rightarrow \{0,1\}$, $ X_1, X_2, \ldots, X_n \in \{0,1\}^d := \underline{2}^d$ be i.i.d.\ random variables drawn from distribution $P_X$, data set $D_n=\{ \left( X_1, f_*(X_1)\right), \ldots, \left( X_n, f_*(X_n)\right) \}$.

Let $f_* \in \mathcal{F} \subset \underline{2}^{ \underline{2}^d } $ and $f \in \underline{2}^{ \underline{2}^d } $. In other words, $\mathcal{X}= \underline{2}^d,\ \mathcal{Y}=\underline{2},\ f_* \in \mathcal{Y}^{\mathcal{X}},\ f: \mathcal{X}\rightarrow \mathcal{Y}$.

Let $P_X^{f_*}:= P(X_1, f_*{X_1})$, and 
\begin{align*}
    & L(f) = \mathbb{P} \left( f(X) \ne f_*(X) \right) = L( P_X^{f_*} ,f ),\\
    & l: \underline{2} \times \underline{2} \rightarrow \underline{2}, \quad l(y,y') = \mathbf{1}(y\ne y'),\\
    & L( P_X^{f_*},f  ) = \int P(\mathrm{d}x, \mathrm{d}y) \ l( f(x),y ).
\end{align*}

\begin{definition}[PAC-Learning]
    Fix $\mathcal{C} = (\mathcal{C}_d)_{d\ge 1}$, where $\mathcal{C}_d \subset \underline{2}^{\underline{2}^d }$. $\mathcal{C}$ is \textbf{PAC-learnable (\underline{P}robably \underline{A}pproximately \underline{C}orrectly)} if $\exists$ polynomial $p\in \mathbb{R}[x,y,z]$ (\textbf{computed with polynomial cost})  and $\mathcal{A}= (\mathcal{A}_{n,d})_{n\ge 1, d\ge 1}$ where $\mathcal{A}_{n,d}:\left( \underline{2}^d\times \underline{2} \right)^n \rightarrow  \underline{2}^{ \underline{2}^d }$
    \begin{align*}
        \text{s.t.\quad }
        & \forall  \varepsilon \in (0,1), \ \delta \in(0,1), \ d\ge 1, \ P \in \mathcal{M}_1(\underline{2}^d),\  f_* \in \mathcal{C}_d,\\
        & n\ge \lceil p( \underbrace{1/\varepsilon}_{\text{accuracy}}, \underbrace{1/\delta}_{\text{confidence}}, d )\rceil,\\
        & X_1,X_2,\ldots,X_n\sim P_X,\\
        & f_n = \mathcal{A}_{n,d} \left( \underbrace{\left( X_1, f_*(X_1)\right), \ldots, \left( X_n, f_*(X_n)\right)}_{D_n} \right)  \quad \text{i.e.\ $D_n \overset{\mathcal{A}}{\rightarrow} f_n$},
    \end{align*}
    we have
    \begin{align*}
        \mathbb{P}\left( L\left( P_X^{f_*},f_n \right) \ge \varepsilon  \right) \le \delta.
    \end{align*}
    In other words, with probability $1-\delta$, $\mathbb{P} \left( f_n(X) \ne f_*(x) | D_n \right) \le \varepsilon$.
\end{definition} 

\begin{remark}[Example]
        \begin{align*}
            & \mathcal{C}_{\text{AND},d}= \left\{  f:   \underline{2}^d \rightarrow \underline{2}\ | \ \exists u \subset[d],\ \forall x\in \underline{2}^d: f(x) = \min_{j\in u} X_j \right\},
            \\
            & \mathcal{C} = \left( \mathcal{C}_{\text{AND},d}\right)_{d\ge 1}.
        \end{align*}
\end{remark}

\section{ERM: Empirical Risk Minimization}

Let
\begin{align*}
    & L_n(f) = \frac{1}{n} \sum_{i=1}^n \mathbf{1}\left( f(X_i) \ne Y_i  \right),
    \\& f_n:= \argmin_{f\in \mathcal{C}_d} L_n(f).
\end{align*}
\underline{Homework:} Show $f_n \argmin_{f\in \mathcal{C}_d} L_n(f)$ is computationally efficient.

Moreover,
\begin{align*}
    f_n:= \argmin_{f\in \mathcal{C}_{\text{AND}}} L_n(f)  \qquad \longrightarrow \qquad \text{proper learning.}
\end{align*}

\noindent
\textbf{Method I:}
Fix $d\ge 1$, $P$ and $f_*$. Let $D_n\rightarrow f_n$. We first decompose $L(f_n)$ as follows:
\begin{align*}
    L(f_n)
    & = L(f_n) - L_n(f_n) + L_n(f_n)
    \\& = \underbrace{ L(f_n) - L_n(f_n) }_{\text{bounded with concentration inequality}}
        + \underbrace{ L_n(f_n) - L_n(f_*) }_{\text{ERM}}
        + \underbrace{ L_n(f_*) - L(f_*)  }_{\text{bounded with concentration inequality}}
        + \underline{L(f_*)}_{=0}.
\end{align*}
Next, Hoeffding inequality implies that with probability $1-\delta$,
\begin{align*}
    L_n(f_*) - L(f_*) \le  \sqrt{\frac{ \log(1/\delta) }{2n} }.
\end{align*}
Besides, $f_n\in \mathcal{C}_d$ indicates that
\begin{align*}
    L(f_n) - L_n(f_n) \le \max_{ f\in \mathcal{C}_d } L(f) - L_n(f_n).
\end{align*}

Fix $f\in \mathcal{C}_d$. Set
\begin{align*}
    \mathcal{U}(f,\delta) = \left\{ L(f) - L_n(f) \le \sqrt{ \frac{\log(1/\delta)}{2n} } \right\}.
\end{align*}
Then
\begin{align*}
    \mathbb{P}\left( \mathcal{U}(f,\delta)  \right) \ge 1- \delta \quad \Longleftrightarrow \quad 
     \mathbb{P}\left( \mathcal{U}^{\text{c}}(f,\delta)  \right) \le \delta.
\end{align*}
Let $N = \left| \mathcal{C}_{\text{AND},d} \right|$ and define `nice event'
\begin{align*}
    \mathcal{U} = \bigcap_{f\in \mathcal{C}_d} \mathcal{U}\left(  f, \frac{\delta}{N} \right).
\end{align*}
Then
\begin{align*}
    \mathbb{P}(\mathcal{U}^{\text{c}})
    = \mathbb{P}\left(  \bigcup_{f\in \mathcal{C}_d} \mathcal{U}^{\text{c}} \left( f, \frac{\delta}{N} \right)  \right)
    \le \sum_{f\in \mathcal{C}_d}  \mathbb{P}\left( \mathcal{U}^{\text{c}} \left( f, \frac{\delta}{N} \right)  \right)
    \le \sum_{f\in \mathcal{C}_d} \frac{ \delta}{N} = \delta. 
\end{align*}

When $\mathcal{U}$ holds, $ L(f) - L_n(f) \le \sqrt{ \frac{\log(N/\delta)}{2n} }$ for all $ f\in \mathcal{C}_d$, in other words,
\begin{align*}
    \max_{ f\in \mathcal{C}_d }  L(f) - L_n(f) \le \sqrt{ \frac{\log(N/\delta)}{2n} }.
\end{align*}

\begin{theorem}[Proper learning]
    $\mathcal{C}_{\text{AND},d}$ PAC-learnable, $f_n$ minimizing the empirical risk, and proper learning: with propability $1-\delta$,
    \begin{align*}
        L(f_n) \le \sqrt{ \frac{\log(N+1/\delta)}{2n} } + \sqrt{ \frac{\log(N+1/\delta)}{n} }.
    \end{align*}
\end{theorem}

\begin{remark}
    \begin{enumerate}[(a)]
        \item This bound on $L(f_n)$ may \textbf{not} be \textbf{tight}.
        \item This result shows PAC-learnability:
        \begin{align*}
            & 2 \sqrt{ \frac{\log(N+1/\delta)}{2n} } \le \varepsilon
            \\ \Leftrightarrow~ &
            \frac{n}{ 2\log( N+1 /\delta ) } \ge \frac{1}{ \varepsilon^2}
            \\ \Leftrightarrow~ &
            n \ge \frac{2}{ \varepsilon^2 } \log\left( \frac{N+1}{\delta} \right).
        \end{align*}
        Hence, $p(1/\varepsilon, 1/\delta, d )  = 2\log\left((|\mathcal{C}_{\text{AND},d}|+1)/\delta \right)/\varepsilon^2$. Since $|\mathcal{C}_{\text{AND},d}|=2^d$, we have
        \begin{align*}
            p\left( \frac{1}{\varepsilon}, \frac{1}{\delta}, d \right) = \frac{ 2\log\left(2^d+1\right) + 2\log(1/\delta) }{\varepsilon^2} \le \ldots
        \end{align*}
    \end{enumerate}
\end{remark}



\noindent
\textbf{Method II:} $L(f) - L_n(f) \le ?$

Fix $0\le \delta \le 1$. By multiplicative Chernoff inequality, with probability $1 - \delta N/(N+1)$, 
\begin{align*}
    L(f) - L_n(f) \le \sqrt{ \frac{ 2L(f) \log(N+1/\delta ) }{ n } } \quad \forall f;
\end{align*}
with probability $1-\delta/(N+1)$,
\begin{align*}
    L_n(f_*) - L(f_*) \le \sqrt{ \frac{ 2L(f_*) \log(N+1/\delta ) }{ n } } + \frac{ \log((N+1)/\delta) }{3n}.
\end{align*}
Denote `nice event'
\begin{align*}
    \mathcal{U}:= \left\{  L(f) - L_n(f) \le \sqrt{ \frac{ 2L(f) \log(N+1/\delta ) }{ n } } \quad \forall f,\   L_n(f_*) - L(f_*) \le \sqrt{ \frac{ 2L(f_*) \log(N+1/\delta ) }{ n } } + \frac{ \log((N+1)/\delta) }{3n} \right\}.
\end{align*}
Then $\mathbb{P}(\mathcal{U}) \ge 1-\delta$.
On $\mathcal{U} \cap \{ L(f_n) \ne 0 \}$: since $L_n(f_n) \ge 0$ and
\begin{align*}
    &
    \frac{ L(f_n) - L_n(f_n) }{ \sqrt{ \frac{ 2L(f_n) \log\left(\frac{N+1}{\delta}\right) }{n} } }
    \le \max_{ f\in \mathcal{C}_d, L(f)\ne 0 }
    \frac{ L(f) - L_n(f) }{ \sqrt{ \frac{ 2L(f) \log\left(\frac{N+1}{\delta}\right) }{n} } } \le 1,
\end{align*}
we have
\begin{align*}
    L(f_n) \le \sqrt{ \frac{ 2L(f_n) \log\left(\frac{N+1}{\delta}\right) }{n} }.
\end{align*}
Furthermore, we have
\begin{align*}
    & L^2(f_n) \le \frac{ 2L(f_n) \log\left(\frac{N+1}{\delta}\right) }{n} ,
    \\& L(f_n) \le \frac{ 2 \log\left(\frac{N+1}{\delta}\right) }{n} \le \varepsilon,
    \\ \Rightarrow~ &
    n \ge \frac{  2\log\left( \frac{ |\mathcal{C}_{d}|+1 }{\delta} \right) }{\varepsilon},
    \\ \Rightarrow~ &
    p\left( \frac{1}{\varepsilon}, \frac{1}{\delta}, d \right) = \ldots
\end{align*}



\end{document}
