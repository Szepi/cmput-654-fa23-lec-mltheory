\documentclass[twoside]{article}

% We add packages, macros here:
%!TEX root =  lec-template.tex
\usepackage{lmodern}
\usepackage[english]{babel}
\usepackage{latexsym}
\usepackage{amsmath}
\usepackage{mathrsfs}
\usepackage{amssymb}
\usepackage{mathtools}
\usepackage[inline,shortlabels]{enumitem} % we prefer enumitem because of its margin adjustment caps
\usepackage{bm}
\usepackage{datetime}
\usepackage[table,xcdraw]{xcolor}
\usepackage{accents}
\usepackage{tikz}
\usepackage{listings}
\usepackage{mdframed}
\usepackage{pgfplots}
\usepackage{pgfplotstable}
\usepackage[boxed]{algorithm}
\usepackage{algpseudocode}
\usepackage{dsfont}
\usepackage{color}
\usepackage{colortbl}
\usepackage{pifont}
\usepackage[bf,font=small,singlelinecheck=off]{caption}
\usepackage{microtype} % improved spacing between words for easier reading
\usepackage{float}
\usepackage{xfrac} % sfrac
\usepackage{xspace}

\linespread{1.1}


\usepackage[textsize=tiny]{todonotes}
\makeatletter
\renewcommand{\todo}[2][]{\@todo[#1]{#2}}
\makeatother

\setlength{\marginparwidth}{10ex}
\newcommand{\todoc}[2][]{\todo[size=\scriptsize,color=blue!20!white,#1]{Csaba: #2}}
\newcommand{\todoj}[2][]{\todo[size=\scriptsize,color=red!20!white,#1]{Jincheng: #2}}

\usepackage{hyperref}
\hypersetup{
    unicode=false,          % non-Latin characters in Acrobat�s bookmarks
    pdftoolbar=true,        % show Acrobat�s toolbar?
    pdfmenubar=true,        % show Acrobat�s menu?
    pdffitwindow=false,     % window fit to page when opened
    pdfstartview={FitH},    % fits the width of the page to the window
    pdftitle={},    % title
    pdfauthor={},     % author
    pdfsubject={Theory, Machine Learning, Lectures},   % subject of the document
    pdfcreator={},   % creator of the document
    pdfproducer={}, % producer of the document
    pdfkeywords={theory} {machine learning} {lecture notes} {CMPUT 654} {Fall 2023}, % list of keywords
    pdfnewwindow=true,      % links in new window
    colorlinks=true,       % false: boxed links; true: colored links
    linkcolor=blue,          % color of internal links (change box color with linkbordercolor)
    citecolor=blue,        % color of links to bibliography
    filecolor=magenta,      % color of file links
    urlcolor=cyan           % color of external links
}
\usepackage{amsthm}
\usepackage{times}
\usepackage{natbib}
\usepackage{nicefrac}
\usepackage{wrapfig}
\usepackage[capitalize]{cleveref}
\usepackage[nottoc,numbib]{tocbibind}

%\usepackage[bmargin=0.75in]{geometry}
\usepackage[margin=1.1in]{geometry}
\usepackage[normalem]{ulem}


%%%%%%%%%%%%%%%%%%%%%%%%%%%%%%%%
% HYPHENATION
%%%%%%%%%%%%%%%%%%%%%%%%%%%%%%%%

\pretolerance=5000
\tolerance=9000
\emergencystretch=0pt
\righthyphenmin=4
\lefthyphenmin=4


\bibliographystyle{plainnat}

\newcommand{\E}{\mathbb{E}}
\newcommand{\EE}[1]{\E[#1]}
\newcommand{\PP}{\mathbb{P}}
\newcommand{\Prob}[1]{\mathbb{P}(#1)}
\newcommand{\one}[1]{\mathbb{I}\{#1\}}
\newcommand{\Supp}{\operatorname{supp}}
\newcommand{\ip}[1]{\langle #1 \rangle}
\newcommand{\bip}[1]{\left\langle #1 \right\rangle}
\newcommand{\norm}[1]{\|#1\|}
\newcommand{\R}{\mathbb{R}}
\newcommand{\N}{\mathbb{N}}
\newcommand{\cA}{\mathcal{A}}
\newcommand{\cB}{\mathcal{B}}
\newcommand{\cC}{\mathcal{C}}
\newcommand{\cD}{\mathcal{D}}
\newcommand{\cE}{\mathcal{E}}
\newcommand{\cF}{\mathcal{F}}
\newcommand{\cG}{\mathcal{G}}
\newcommand{\cH}{\mathcal{H}}
\newcommand{\cK}{\mathcal{K}}
\newcommand{\cL}{\mathcal{L}}
\newcommand{\cN}{\mathcal{N}}
\newcommand{\cP}{\mathcal{P}}
\newcommand{\cQ}{\mathcal{Q}}
\newcommand{\cR}{\mathcal{R}}
\newcommand{\cS}{\mathcal{S}}
\newcommand{\sA}{\mathscr A}

\newcommand{\epsapp}{\epsilon}
\newcommand{\epssub}{\delta}

\DeclareMathOperator{\Range}{range}
\newcommand{\rows}{\operatorname{rows}}

\renewcommand{\epsilon}{\varepsilon}
\newcommand{\ceil}[1]{\left\lceil {#1} \right\rceil}
\newcommand{\floor}[1]{\left\lfloor {#1} \right\rfloor}
\newcommand{\ones}{\mathbf{1}}
\newcommand{\zeros}{\mathbf{0}}
\DeclareMathOperator*{\argmin}{arg\ min}
\DeclareMathOperator*{\argmax}{arg\ max}


\def\rvzero{{\mathbf{0}}}
\def\rvone{{\mathbf{1}}}

\def\identiymatrix{\mathbf{Id}}

\newcommand{\softmax}{\mathrm{softmax}}
\newcommand{\KL}{D_{\mathrm{KL}}}

% Graph
\def\gA{{\mathcal{A}}}
\def\gB{{\mathcal{B}}}
\def\gC{{\mathcal{C}}}
\def\gD{{\mathcal{D}}}
\def\gE{{\mathcal{E}}}
\def\gF{{\mathcal{F}}}
\def\gG{{\mathcal{G}}}
\def\gH{{\mathcal{H}}}
\def\gI{{\mathcal{I}}}
\def\gJ{{\mathcal{J}}}
\def\gK{{\mathcal{K}}}
\def\gL{{\mathcal{L}}}
\def\gM{{\mathcal{M}}}
\def\gN{{\mathcal{N}}}
\def\gO{{\mathcal{O}}}
\def\gP{{\mathcal{P}}}
\def\gQ{{\mathcal{Q}}}
\def\gR{{\mathcal{R}}}
\def\gS{{\mathcal{S}}}
\def\gT{{\mathcal{T}}}
\def\gU{{\mathcal{U}}}
\def\gV{{\mathcal{V}}}
\def\gW{{\mathcal{W}}}
\def\gX{{\mathcal{X}}}
\def\gY{{\mathcal{Y}}}
\def\gZ{{\mathcal{Z}}}

% Sets
\def\sA{{\mathbb{A}}}
\def\sB{{\mathbb{B}}}
\def\sC{{\mathbb{C}}}
\def\sD{{\mathbb{D}}}
% Don't use a set called E, because this would be the same as our symbol
% for expectation.
\def\sE{{\mathbb{E}}}
\def\sF{{\mathbb{F}}}
\def\sG{{\mathbb{G}}}
\def\sH{{\mathbb{H}}}
\def\sI{{\mathbb{I}}}
\def\sJ{{\mathbb{J}}}
\def\sK{{\mathbb{K}}}
\def\sL{{\mathbb{L}}}
\def\sM{{\mathbb{M}}}
\def\sN{{\mathbb{N}}}
\def\sO{{\mathbb{O}}}
\def\sP{{\mathbb{P}}}
\def\sQ{{\mathbb{Q}}}
\def\sR{{\mathbb{R}}}
\def\sS{{\mathbb{S}}}
\def\sT{{\mathbb{T}}}
\def\sU{{\mathbb{U}}}
\def\sV{{\mathbb{V}}}
\def\sW{{\mathbb{W}}}
\def\sX{{\mathbb{X}}}
\def\sY{{\mathbb{Y}}}
\def\sZ{{\mathbb{Z}}}

\newcommand{\dimE}{\mathrm{dim}_{\mathcal{E}}}
\DeclareMathOperator{\diam}{diam}


%%
%% ADD PACKAGES here:
%%
%
%\usepackage{amsmath,amsfonts,graphicx}
%
%
% The following commands set up the lecnum (lecture number)
% counter and make various numbering schemes work relative
% to the lecture number.
%

\newcounter{lecnum}
\renewcommand{\thepage}{\thelecnum-\arabic{page}}
\renewcommand{\thesection}{\thelecnum.\arabic{section}}
\renewcommand{\theequation}{\thelecnum.\arabic{equation}}
\renewcommand{\thefigure}{\thelecnum.\arabic{figure}}
\renewcommand{\thetable}{\thelecnum.\arabic{table}}


%
% The following macro is used to generate the header.
%
\newcommand{\lecture}[4]{
   \pagestyle{myheadings}
   \thispagestyle{plain}
   \newpage
   \setcounter{lecnum}{#1}
   \setcounter{page}{1}
   \noindent
   \begin{center}
   \framebox{
      \vbox{\vspace{2mm}
    \hbox to 6.28in { {\bf CMPUT 654 Fa 23: Theoretical Foundations of Machine Learning \hfill Fall 2023} }
       \vspace{4mm}
       \hbox to 6.28in { {\Large \hfill Lecture #1: #2  \hfill} }
       \vspace{2mm}
       \hbox to 6.28in { {\it Lecturer: #3 \hfill Scribes: #4} }
      \vspace{2mm}}
   }
   \end{center}
   \markboth{Lecture #1: #2}{Lecture #1: #2}

   \noindent {\bf Note}: {\it 
   \LaTeX\  template courtesy of UC Berkeley EECS dept. (\href{https://inst.eecs.berkeley.edu/~cs294-8/sp03/Materials/}{link} to directory)
   }

   \noindent {\bf Disclaimer}: {\it These notes have \underline{\textbf{not}} been subjected to the
   usual scrutiny reserved for formal publications. They may be
   distributed outside this class only with the permission of the
   Instructor.} \vspace*{4mm}
}
%
% Convention for citations is authors' initials followed by the year.
% For example, to cite a paper by Leighton and Maggs you would type
% \cite{LM89}, and to cite a paper by Strassen you would type \cite{S69}.
%%%%%%%%% (To avoid bibliography problems, for now we redefine the \cite command.)
%%%%%%%%% Also commands that create a suitable format for the reference list.
%%%%%%%%\renewcommand{\cite}[1]{[#1]}
%%%%%%%%\def\beginrefs{\begin{list}%
%%%%%%%%        {[\arabic{equation}]}{\usecounter{equation}
%%%%%%%%         \setlength{\leftmargin}{2.0truecm}\setlength{\labelsep}{0.4truecm}%
%%%%%%%%         \setlength{\labelwidth}{1.6truecm}}}
%%%%%%%%\def\endrefs{\end{list}}
%%%%%%%%\def\bibentry#1{\item[\hbox{[#1]}]}

%Use this command for a figure; it puts a figure in wherever you want it.
%usage: \fig{NUMBER}{SPACE-IN-INCHES}{CAPTION}
\newcommand{\fig}[3]{
			\vspace{#2}
			\begin{center}
			Figure \thelecnum.#1:~#3
			\end{center}
	}
% Use these for theorems, lemmas, proofs, etc.

%!TEX root =  lec-template.tex
%%%%%%%%%%%%%%%%%%%%%%%%%%%%%%%%
% THEOREMS
%%%%%%%%%%%%%%%%%%%%%%%%%%%%%%%%
\theoremstyle{plain}
\newtheorem{theorem}{Theorem}[lecnum]
\newtheorem{claim}[theorem]{Claim}
\newtheorem{proposition}[theorem]{Proposition}
\newtheorem{lemma}[theorem]{Lemma}
\newtheorem{corollary}[theorem]{Corollary}
\newtheorem{example}[theorem]{Example}
\theoremstyle{definition}
\newtheorem{definition}[theorem]{Definition}
\newtheorem{assumption}[theorem]{Assumption}
\newtheorem{remark}[theorem]{Remark}
\newtheorem{exercise}[theorem]{Exercise}
\theoremstyle{remark}


%\newtheorem{theorem}{Theorem}[lecnum]
%\newtheorem{lemma}[theorem]{Lemma}
%\newtheorem{proposition}[theorem]{Proposition}
%\newtheorem{claim}[theorem]{Claim}
%\newtheorem{corollary}[theorem]{Corollary}
%\newtheorem{definition}[theorem]{Definition}
%\newenvironment{proof}{{\bf Proof:}}{\hfill\rule{2mm}{2mm}}

% **** IF YOU WANT TO DEFINE ADDITIONAL MACROS FOR YOURSELF, PUT THEM HERE:

\newcommand\myE{\mathbb{E}}

\begin{document}
%FILL IN THE RIGHT INFO.
%\lecture{**LECTURE-NUMBER**}{**DATE**}{**LECTURER**}{**SCRIBE**}
\lecture{6}{September 11}{Csaba Szepesv\'ari}{Shuai Liu}
%\footnotetext{These notes are partially based on those of Nigel Mansell.}

% **** YOUR NOTES GO HERE:

% Some general latex examples and examples making use of the
% macros follow.  
%**** IN GENERAL, BE BRIEF. LONG SCRIBE NOTES, NO MATTER HOW WELL WRITTEN,
%**** ARE NEVER READ BY ANYBODY.
This lecture's notes illustrate some uses of
various \LaTeX\ macros.  
Take a look at this and imitate.

\section{Recap}
 % Don't be this informal in your notes!
Recall the setting of ERM introduced in the previous lectures. We have a dataset (or datalist) $D_n=\{(X_i, f_*(X_i))\}_{i=1}^n$ where $X_i \sim P\in \mathcal M_1(\mathcal X)$ are independent and $f_*\in C_d\subset \underline{2}^{\underline{2}^d}$. Let $|C_d|=N<\infty$. For a fixed function $f\in \underline{2}^{\underline{2}^d}$, let $L_n(f)=\sum_{i=1}^n \mathbb I(f(X_i)\neq f_*(Y_i))$ and $L(f)=\mathbb E[\mathbb I(f(X)\neq f_*(X))]$ for $X\sim P$. The empirical risk minimizer is $f_n = \arg\min_{f\in C_d}L_n(f)$. We used the multiplicative Chernoff bound to obtain the following proposition:
\begin{proposition}\label{prop:recap}
   For $\delta\in (0,1)$, $f\in \underline{2}^{\underline{2}^d}$ and $n,N\in \mathbb N$, let $\beta^n_{\delta}(f,N)=\sqrt{\frac{2L(f)\log(\frac{N}{\delta})}{n}}$. 
   For all $f_0\in C_d$ and $\delta\in (0,1)$, let $U(\delta,f_0,C_d)$ be the event that:
   \begin{align*}
      U(\delta,f_0, C_d):=\Bigg\{\forall f\in C_d: L(f)\le L_n(f)+ \beta^n_{\delta}(f,N+1)\Bigg\}\bigcap\left\{L_n(f_0)\le L(f_0)+\beta^n_{\delta}(f_0,N+1)+\frac{\log(\frac{N+1}{\delta})}{3n}\right\}.
   \end{align*}
   It follows that $\mathbb P(U(\delta,f_0, C_d))\ge 1-\delta$.
\end{proposition}
For all $f_0\in C_d$, on the event $U(\delta,f_0, C_d)$, we have that:
\begin{align*}
   L(f_n)&\le L_n(f_n)+\beta^n_{\delta}(f_n,N+1)\\
   &\le L_n(f_0)+\beta^n_{\delta}(f_n,N+1)\tag{$f_n$ is the sol. to ERM}\\
   &\le L(f_0)+\beta^n_{\delta}(f_0,N+1)+\beta^n_{\delta}(f_n,N+1)+\frac{\log(\frac{N+1}{\delta})}{3n},
\end{align*}
which gives us the following theorem:
\begin{theorem}
   For all $f_0\in C_d$, w.p. $1-\delta$,
   \begin{equation*}
      L(f_n)\le L(f_0)+\beta^n_{\delta}(f_0,N+1)+\beta^n_{\delta}(f_n,N+1)+\frac{\log(\frac{N+1}{\delta})}{3n}.
   \end{equation*}
\end{theorem}
Since the above theorem holds for all $f_0\in C_d$, we can take the infimum:
\begin{corollary}\label{coro:inf}
   w.p. $1-\delta$,
   \begin{equation*}
      L(f_n)\le \beta^n_{\delta}(f_n,N+1)+\frac{\log(\frac{N+1}{\delta})}{3n}+\inf_{f\in C_d(\delta)}\left( L(f)+\beta^n_{\delta}(f,N+1)\right)
   \end{equation*}
\end{corollary}
\begin{remark}
   In our current setting, $\inf_{f\in C_d(\delta)}(L(f)+\beta^n_{\delta}(f,N+1)) = 0$ because $L(f_*)+\beta^n_{\delta}(f_*,N+1)=0$. \cref{coro:inf} cannot buy us anything more than the bound we got in the last class because there is still a factor of $\sqrt{1/n}$ in $\beta_\delta^n(f_n,N+1)$. However, in more general settings where $L(f_*)\neq 0$, i.e., noises are injected to $f_*(X_i)$, we may get some benefit from \cref{coro:inf}.
\end{remark}
\section{Empirical Process}
Now consider an arbitrary function class $\mathcal F\subset \mathcal Y^\mathcal X$ which is potentially infinite and an arbitrary (measurable) loss function $\ell:\mathcal Y\times\mathcal Y \to \mathbb R$ (instead of the $0$-$1$ loss we considered in the previous section). Let $f_n=\arg\max_{f\in \mathcal F}L_n(f)$ be the empirical risk minimizer on $\mathcal F$. If we were to apply the technique in \cref{prop:recap}, the term $L_n(f) - L(f)$ for some $f\in \mathcal F$, would be the quantity that we would like to bound. To do that, one of the options is to bound:
\begin{equation}
   \sup_{f\in \mathcal F}|L_n(f)-L(f)|=\sup_{f\in \mathcal F}\frac{1}{n} \left|\sum_{i=1}^n\ell(f(X_i),Y_i) - \int\ell(f(x),y)P(dx,dy)\right|\label{eq:uniform_boundedness}
\end{equation}
To reduce clutter, we define $D_i:\mathcal F\to\mathbb R$ for $i\in \mathbb N$ such that
\begin{equation*}
   D_i(f) = \ell(f(X_i),Y_i) - \int\ell(f(x),y)P(dx,dy),
\end{equation*}
and $\bar D_n:\mathcal F\to\mathbb R$ such that
\begin{equation*}
    \bar D_n(f) = \frac{1}{n}\sum_{i=1}^n D_i(f),\quad \forall f\in \mathcal F.
\end{equation*}
Note that $D_1(f), D_2(f),...$ are i.i.d. random variables. Then \cref{eq:uniform_boundedness} can be written as:
\begin{equation*}
   \sup_{f\in \mathcal F}\bar D_n(f).
\end{equation*}
We call $\{\bar D_n(f)\}_{n=1}^\infty$ an empirical process. Empirical process theory is a subarea of probability theory that studies the question of convergence of the process to $0$ in different ways, e.g., convergence in probability or almost sure convergence. If $\bar D_n(f)\to 0$ \href{https://en.wikipedia.org/wiki/Convergence_of_random_variables#Convergence_in_probability}{in probability}, it is called the \textit{Weak Law of Large Number} and when $\sup_{f\in \mathcal F}\bar D_n(f)\to 0$ happens, we say that \textit{uniform convergence} happens.
\section{Lower Bracketing Number}
Now we further reduce the clutter by introducing new notations. Let $\mathcal Z=\mathcal X\times \mathcal Y$
\begin{equation*}
   G = \{(x,y)\to \ell(f(x),y):f\in \mathcal F\}\subseteq \mathbb R^{\mathcal X\times\mathcal Y}=\mathbb R^{\mathcal Z}.
\end{equation*}
Let $Z_1,Z_2,...Z_n\sim P\in \mathcal M_1(\mathcal Z)$ and let $P_n(dz)=\frac{1}{n}\sum_{i=1}^n \delta_{Z_i}(dz)$ be the \textit{empirical distribution} where $\delta_{Z_i}(\{z\})=1$ if $z=Z_i$ and $0$ otherwise. Note that $\delta_{Z_i}$ is a random measure. For $P\in\mathcal M_1(\mathcal Z)$, let $Pg:=\int gdP$ for $g\in \mathcal G$. Then \cref{eq:uniform_boundedness} can be written as:
\begin{equation*}
   \sup_{g\in \mathcal G}\left|P_ng - Pg\right|
\end{equation*}
\begin{definition}\label{defn:lower_bracketing_cover}
   Let $\mathcal G \subseteq \mathbb R^\mathcal Z$ and fix $P\in \mathcal M_1(\mathcal Z)$. For a fixed $\varepsilon$, $g_1,...g_m\in \mathbb R^\mathcal Z$ is called a lower bracketing cover of $\mathcal G\text{@}P\text{@}\varepsilon$ if for all $g\in \mathcal G$, there exists $j\in [m]$ such that:
   \begin{enumerate}
      \item $g_j\le g$,
      \item $Pg\le Pg_j+\varepsilon$.
   \end{enumerate}
   Note that $g_1,...,g_m$ is not necessarily in $\mathcal G$.
\end{definition}
\begin{theorem}
   Let $\mathcal G\subset [0,1]^\mathcal Z$, $P\in \mathcal M_1(\mathcal Z)$ and $Z_1,...,Z_n\sim P$ for $n\in \mathbb N$. For all $\varepsilon>0$, $\delta\in (0,1)$ and $g\in \mathcal G$, it follows that w.p. $1-\delta$,
   \begin{equation*}
      Pg\le P_ng+\inf_{\varepsilon>0}\left[\varepsilon+\sqrt{\frac{\log(N_\varepsilon/\delta)}{2n}}\right],
   \end{equation*}
   where for all $\varepsilon>0$, 
   \begin{align*}
      N_\varepsilon =\min\{n\in \mathbb N: \text{ there exists }g_1,...,g_n \text{ such that }(g_1,...,g_n) \text{ is a lower bracketing cover of }\mathcal G\text{@}P\text{@}\varepsilon\}
   \end{align*}
\end{theorem}
\begin{proof}
   Fix an $\varepsilon>0$. Let $m=N_\varepsilon$ and $g_1,...,g_m$ be $\text{a lower bracketing cover of }\mathcal G\text{@}P\text{@}\varepsilon$. Using additive Chernoff bound, we have that w.p. at least $1-\delta$, it follows that
   \begin{equation}
      Pg_j\le P_ng_j+\sqrt{\frac{\log(N_\varepsilon/\delta)}{2n}}\label{eq:additive_chernoff}. 
   \end{equation}
   Pick $g\in \mathcal G$ and by definition of lower bracketing cover, there exists $j\in [m]$ such that
   \begin{align*}
      Pg\le Pg_j+\varepsilon&\le P_ng_j + \varepsilon + \sqrt{\frac{\log(N_\varepsilon/\delta)}{2n}}\tag{\cref{defn:lower_bracketing_cover}(1) and \cref{eq:additive_chernoff}}\\
      &\le P_ng+\varepsilon+\sqrt{\frac{\log(N_\varepsilon/\delta)}{2n}}\tag{\cref{defn:lower_bracketing_cover}(2)}.
   \end{align*}
   Since $\varepsilon$ was arbitrary, we then take the infimum over $\varepsilon$:
   \begin{equation*}
      Pg\le P_ng+\inf_{\varepsilon>0}\left[\varepsilon+\sqrt{\frac{\log(N_\varepsilon/\delta)}{2n}}\right].
   \end{equation*}
\end{proof}

% \bibliography{all}
% **** THIS ENDS THE EXAMPLES. DON'T DELETE THE FOLLOWING LINE:

\end{document}





