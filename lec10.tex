\documentclass[twoside]{article}

% We add packages, macros here:
%!TEX root =  lec-template.tex
\usepackage{lmodern}
\usepackage[english]{babel}
\usepackage{latexsym}
\usepackage{amsmath}
\usepackage{mathrsfs}
\usepackage{amssymb}
\usepackage{mathtools}
\usepackage[inline,shortlabels]{enumitem} % we prefer enumitem because of its margin adjustment caps
\usepackage{bm}
\usepackage{datetime}
\usepackage[table,xcdraw]{xcolor}
\usepackage{accents}
\usepackage{tikz}
\usepackage{listings}
\usepackage{mdframed}
\usepackage{pgfplots}
\usepackage{pgfplotstable}
\usepackage[boxed]{algorithm}
\usepackage{algpseudocode}
\usepackage{dsfont}
\usepackage{color}
\usepackage{colortbl}
\usepackage{pifont}
\usepackage[bf,font=small,singlelinecheck=off]{caption}
\usepackage{microtype} % improved spacing between words for easier reading
\usepackage{float}
\usepackage{xfrac} % sfrac
\usepackage{xspace}

\linespread{1.1}


\usepackage[textsize=tiny]{todonotes}
\makeatletter
\renewcommand{\todo}[2][]{\@todo[#1]{#2}}
\makeatother

\setlength{\marginparwidth}{10ex}
\newcommand{\todoc}[2][]{\todo[size=\scriptsize,color=blue!20!white,#1]{Csaba: #2}}
\newcommand{\todoj}[2][]{\todo[size=\scriptsize,color=red!20!white,#1]{Jincheng: #2}}

\usepackage{hyperref}
\hypersetup{
    unicode=false,          % non-Latin characters in Acrobat�s bookmarks
    pdftoolbar=true,        % show Acrobat�s toolbar?
    pdfmenubar=true,        % show Acrobat�s menu?
    pdffitwindow=false,     % window fit to page when opened
    pdfstartview={FitH},    % fits the width of the page to the window
    pdftitle={},    % title
    pdfauthor={},     % author
    pdfsubject={Theory, Machine Learning, Lectures},   % subject of the document
    pdfcreator={},   % creator of the document
    pdfproducer={}, % producer of the document
    pdfkeywords={theory} {machine learning} {lecture notes} {CMPUT 654} {Fall 2023}, % list of keywords
    pdfnewwindow=true,      % links in new window
    colorlinks=true,       % false: boxed links; true: colored links
    linkcolor=blue,          % color of internal links (change box color with linkbordercolor)
    citecolor=blue,        % color of links to bibliography
    filecolor=magenta,      % color of file links
    urlcolor=cyan           % color of external links
}
\usepackage{amsthm}
\usepackage{times}
\usepackage{natbib}
\usepackage{nicefrac}
\usepackage{wrapfig}
\usepackage[capitalize]{cleveref}
\usepackage[nottoc,numbib]{tocbibind}

%\usepackage[bmargin=0.75in]{geometry}
\usepackage[margin=1.1in]{geometry}
\usepackage[normalem]{ulem}


%%%%%%%%%%%%%%%%%%%%%%%%%%%%%%%%
% HYPHENATION
%%%%%%%%%%%%%%%%%%%%%%%%%%%%%%%%

\pretolerance=5000
\tolerance=9000
\emergencystretch=0pt
\righthyphenmin=4
\lefthyphenmin=4


\bibliographystyle{plainnat}

\newcommand{\E}{\mathbb{E}}
\newcommand{\EE}[1]{\E[#1]}
\newcommand{\PP}{\mathbb{P}}
\newcommand{\Prob}[1]{\mathbb{P}(#1)}
\newcommand{\one}[1]{\mathbb{I}\{#1\}}
\newcommand{\Supp}{\operatorname{supp}}
\newcommand{\ip}[1]{\langle #1 \rangle}
\newcommand{\bip}[1]{\left\langle #1 \right\rangle}
\newcommand{\norm}[1]{\|#1\|}
\newcommand{\R}{\mathbb{R}}
\newcommand{\N}{\mathbb{N}}
\newcommand{\cA}{\mathcal{A}}
\newcommand{\cB}{\mathcal{B}}
\newcommand{\cC}{\mathcal{C}}
\newcommand{\cD}{\mathcal{D}}
\newcommand{\cE}{\mathcal{E}}
\newcommand{\cF}{\mathcal{F}}
\newcommand{\cG}{\mathcal{G}}
\newcommand{\cH}{\mathcal{H}}
\newcommand{\cK}{\mathcal{K}}
\newcommand{\cL}{\mathcal{L}}
\newcommand{\cN}{\mathcal{N}}
\newcommand{\cP}{\mathcal{P}}
\newcommand{\cQ}{\mathcal{Q}}
\newcommand{\cR}{\mathcal{R}}
\newcommand{\cS}{\mathcal{S}}
\newcommand{\sA}{\mathscr A}

\newcommand{\cM}{\mathcal{M}}
\newcommand{\cX}{\mathcal{X}}
\newcommand{\cY}{\mathcal{Y}}
\newcommand{\NN}{\mathbb{N}}
\newcommand{\RR}{\mathbb{R}}
\newcommand{\VV}[1]{\mathbb{V}[#1]}

\newcommand{\epsapp}{\epsilon}
\newcommand{\epssub}{\delta}

\DeclareMathOperator{\Range}{range}
\newcommand{\rows}{\operatorname{rows}}

\renewcommand{\epsilon}{\varepsilon}
\newcommand{\ceil}[1]{\left\lceil {#1} \right\rceil}
\newcommand{\floor}[1]{\left\lfloor {#1} \right\rfloor}
\newcommand{\ones}{\mathbf{1}}
\newcommand{\zeros}{\mathbf{0}}
\DeclareMathOperator*{\argmin}{arg\ min}
\DeclareMathOperator*{\argmax}{arg\ max}


\def\rvzero{{\mathbf{0}}}
\def\rvone{{\mathbf{1}}}

\def\identiymatrix{\mathbf{Id}}

\newcommand{\softmax}{\mathrm{softmax}}
\newcommand{\KL}{D_{\mathrm{KL}}}

% Graph
\def\gA{{\mathcal{A}}}
\def\gB{{\mathcal{B}}}
\def\gC{{\mathcal{C}}}
\def\gD{{\mathcal{D}}}
\def\gE{{\mathcal{E}}}
\def\gF{{\mathcal{F}}}
\def\gG{{\mathcal{G}}}
\def\gH{{\mathcal{H}}}
\def\gI{{\mathcal{I}}}
\def\gJ{{\mathcal{J}}}
\def\gK{{\mathcal{K}}}
\def\gL{{\mathcal{L}}}
\def\gM{{\mathcal{M}}}
\def\gN{{\mathcal{N}}}
\def\gO{{\mathcal{O}}}
\def\gP{{\mathcal{P}}}
\def\gQ{{\mathcal{Q}}}
\def\gR{{\mathcal{R}}}
\def\gS{{\mathcal{S}}}
\def\gT{{\mathcal{T}}}
\def\gU{{\mathcal{U}}}
\def\gV{{\mathcal{V}}}
\def\gW{{\mathcal{W}}}
\def\gX{{\mathcal{X}}}
\def\gY{{\mathcal{Y}}}
\def\gZ{{\mathcal{Z}}}

% Sets
\def\sA{{\mathbb{A}}}
\def\sB{{\mathbb{B}}}
\def\sC{{\mathbb{C}}}
\def\sD{{\mathbb{D}}}
% Don't use a set called E, because this would be the same as our symbol
% for expectation.
\def\sE{{\mathbb{E}}}
\def\sF{{\mathbb{F}}}
\def\sG{{\mathbb{G}}}
\def\sH{{\mathbb{H}}}
\def\sI{{\mathbb{I}}}
\def\sJ{{\mathbb{J}}}
\def\sK{{\mathbb{K}}}
\def\sL{{\mathbb{L}}}
\def\sM{{\mathbb{M}}}
\def\sN{{\mathbb{N}}}
\def\sO{{\mathbb{O}}}
\def\sP{{\mathbb{P}}}
\def\sQ{{\mathbb{Q}}}
\def\sR{{\mathbb{R}}}
\def\sS{{\mathbb{S}}}
\def\sT{{\mathbb{T}}}
\def\sU{{\mathbb{U}}}
\def\sV{{\mathbb{V}}}
\def\sW{{\mathbb{W}}}
\def\sX{{\mathbb{X}}}
\def\sY{{\mathbb{Y}}}
\def\sZ{{\mathbb{Z}}}

\newcommand{\dimE}{\mathrm{dim}_{\mathcal{E}}}
\DeclareMathOperator{\diam}{diam}


%%
%% ADD PACKAGES here:
%%
%
%\usepackage{amsmath,amsfonts,graphicx}
%
%
% The following commands set up the lecnum (lecture number)
% counter and make various numbering schemes work relative
% to the lecture number.
%

\newcounter{lecnum}
\renewcommand{\thepage}{\thelecnum-\arabic{page}}
\renewcommand{\thesection}{\thelecnum.\arabic{section}}
\renewcommand{\theequation}{\thelecnum.\arabic{equation}}
\renewcommand{\thefigure}{\thelecnum.\arabic{figure}}
\renewcommand{\thetable}{\thelecnum.\arabic{table}}


%
% The following macro is used to generate the header.
%
\newcommand{\lecture}[4]{
   \pagestyle{myheadings}
   \thispagestyle{plain}
   \newpage
   \setcounter{lecnum}{#1}
   \setcounter{page}{1}
   \noindent
   \begin{center}
   \framebox{
      \vbox{\vspace{2mm}
    \hbox to 6.28in { {\bf CMPUT 654 Fa 23: Theoretical Foundations of Machine Learning \hfill Fall 2023} }
       \vspace{4mm}
       \hbox to 6.28in { {\Large \hfill Lecture #1: #2  \hfill} }
       \vspace{2mm}
       \hbox to 6.28in { {\it Lecturer: #3 \hfill Scribes: #4} }
      \vspace{2mm}}
   }
   \end{center}
   \markboth{Lecture #1: #2}{Lecture #1: #2}

   \noindent {\bf Note}: {\it 
   \LaTeX\  template courtesy of UC Berkeley EECS dept. (\href{https://inst.eecs.berkeley.edu/~cs294-8/sp03/Materials/}{link} to directory)
   }

   \noindent {\bf Disclaimer}: {\it These notes have \underline{\textbf{not}} been subjected to the
   usual scrutiny reserved for formal publications. They may be
   distributed outside this class only with the permission of the
   Instructor.} \vspace*{4mm}
}
%
% Convention for citations is authors' initials followed by the year.
% For example, to cite a paper by Leighton and Maggs you would type
% \cite{LM89}, and to cite a paper by Strassen you would type \cite{S69}.
%%%%%%%%% (To avoid bibliography problems, for now we redefine the \cite command.)
%%%%%%%%% Also commands that create a suitable format for the reference list.
%%%%%%%%\renewcommand{\cite}[1]{[#1]}
%%%%%%%%\def\beginrefs{\begin{list}%
%%%%%%%%        {[\arabic{equation}]}{\usecounter{equation}
%%%%%%%%         \setlength{\leftmargin}{2.0truecm}\setlength{\labelsep}{0.4truecm}%
%%%%%%%%         \setlength{\labelwidth}{1.6truecm}}}
%%%%%%%%\def\endrefs{\end{list}}
%%%%%%%%\def\bibentry#1{\item[\hbox{[#1]}]}

%Use this command for a figure; it puts a figure in wherever you want it.
%usage: \fig{NUMBER}{SPACE-IN-INCHES}{CAPTION}
\newcommand{\fig}[3]{
			\vspace{#2}
			\begin{center}
			Figure \thelecnum.#1:~#3
			\end{center}
	}
% Use these for theorems, lemmas, proofs, etc.

%!TEX root =  lec-template.tex
%%%%%%%%%%%%%%%%%%%%%%%%%%%%%%%%
% THEOREMS
%%%%%%%%%%%%%%%%%%%%%%%%%%%%%%%%
\theoremstyle{plain}
\newtheorem{theorem}{Theorem}[lecnum]
\newtheorem{claim}[theorem]{Claim}
\newtheorem{proposition}[theorem]{Proposition}
\newtheorem{lemma}[theorem]{Lemma}
\newtheorem{corollary}[theorem]{Corollary}
\newtheorem{example}[theorem]{Example}
\theoremstyle{definition}
\newtheorem{definition}[theorem]{Definition}
\newtheorem{assumption}[theorem]{Assumption}
\newtheorem{remark}[theorem]{Remark}
\newtheorem{exercise}[theorem]{Exercise}
\theoremstyle{remark}


%\newtheorem{theorem}{Theorem}[lecnum]
%\newtheorem{lemma}[theorem]{Lemma}
%\newtheorem{proposition}[theorem]{Proposition}
%\newtheorem{claim}[theorem]{Claim}
%\newtheorem{corollary}[theorem]{Corollary}
%\newtheorem{definition}[theorem]{Definition}
%\newenvironment{proof}{{\bf Proof:}}{\hfill\rule{2mm}{2mm}}

% **** IF YOU WANT TO DEFINE ADDITIONAL MACROS FOR YOURSELF, PUT THEM HERE:

\newcommand{\set}[1]{\left\{#1\right\}}
\newcommand{\I}{\mathbb{I}}
\newcommand{\Fc}{\cF_\varepsilon(Z_{1:n})}


\begin{document}
%FILL IN THE RIGHT INFO.
% \lecture{**LECTURE-NUMBER**}{**DATE**}{**LECTURER**}{**SCRIBE**}
\lecture{10}{Empirical Covering Number Bounds (October 5)}{Csaba Szepesv\'ari}{Kushagra Chandak, Shuai Liu}
%\footnotetext{These notes are partially based on those of Nigel Mansell.}

\newcommand{\Rad}{\mathrm{Rad}}

% **** YOUR NOTES GO HERE:

% Some general latex examples and examples making use of the
% macros follow.  
%**** IN GENERAL, BE BRIEF. LONG SCRIBE NOTES, NO MATTER HOW WELL WRITTEN,
%**** ARE NEVER READ BY ANYBODY.
In this lecture we are going to prove a theorem using symmetrization lemma (and some other lemmas of course). This theorem talks about how one is able to achieve uniform convergence with empirical $L_1$ cover. We first state the theorem here and defer the proof to later.
\begin{theorem}\label{thm:unif_L1}
    Let $P\in \cM_1(\cZ)$ and $\cG\subseteq [0,1]^\cZ$. Then for all $\delta\in (0,1)$, with probability at least $1-\delta$, it follows that for all $g\in \cG$,
    \begin{equation*}
        Pg\le P_ng+2\varepsilon+3\sqrt{\frac{\log (2N_1+1)/\delta}{2n}},
    \end{equation*}
    where $N_1:=N_1(\varepsilon, \cG, n)$.
\end{theorem}
For infinite function classes, earlier we had uniform one-sided deviation bounds based on bracketing numbers. As discussed earlier, bracketing numbers do not give good bounds for classifications. So we are going to replace them with $L_1$ uniform covering numbers. Before we state the theorem, we will need another lemma apart from the symmetrization lemma:
\begin{lemma}\label{lem:psi}
    Let $P \in \cM_1(\cZ)$ be a distribution over $\cZ$, $(Z, Z') \sim P^{\otimes 2n}$ (where we omit the subscript $1:n$ for both $Z,Z'$) and $\cF$ be the function class that we are considering. Assume that the following conditions hold for $\psi:\cF \times \cZ^n \to \R$, $\psi':\cF \times \cZ^n \to \R$, $\psi:\cF \to \R$ and $0 < \delta_1, \delta_2 < 1$, $\varepsilon > 0$.
    \begin{enumerate}
        \item (U) w.p. $1-\delta_1$, $\forall f$, $\psi'(f, Z') \le \psi(f,Z)$.
        \item (NU) $\forall f$, w.p. $1-\delta_2$, $\psi(f) \le \psi'(f, Z') + \varepsilon$.
    \end{enumerate}
    Then w.p. $1-\delta_1-\delta_2$, it holds that for all $ f\in \cF$,
    \[
        \psi(f) \le \psi(f,Z) + \varepsilon\,.
    \]
    Note that there is an overloading of notation $\psi$ here but it should be obvious from the context.
\end{lemma}
\begin{proof}
    Define $Q(f,z)=\psi(f)-\psi(f,z)-\varepsilon$ for $z\in \cZ^n$. WLOG assume the existence of 
    \[\hat f(z)\in \arg\max_{f\in \cF}Q(f,z),\]
    where $f(z)$ indicates the dependence of $f$ on $z$ (if it does not exist, we can always approximate the supremum with arbitrary precision $\varepsilon'$ and push it to $0$ at the end).
    Let $\cE:=\{\forall f\in \cF:Q(f,Z)\le 0\}$. It then suffices to show that $\PP(\cE)\ge 1-\delta_1-\delta_2$.
    Note that on the event $\cE$, it holds that $Q(\hat f(Z),Z)\le 0$ where we apply the definition of $\hat f$. Similarly, on $\cE^c$, we have that $Q(\hat f(Z),Z)>0$ hence $\cE$ is equivalent to $\{Q(\hat f(Z),Z)\le 0\}:=\tilde \cE$. By assumption $(U)$, $\psi'(\hat f(Z),Z')\le \psi (\hat f(Z),Z)$ holds w.p. $1-\delta_2$. By $(NU)$, w.p. $1-\delta_2$, we have that $\psi(\hat f(Z))\le \psi'(\hat f(Z),Z')+\varepsilon$. Note that we cannot directly conclude that out of rigorousness as the assumption (NU) only states that for a fixed function. An extra step is needed, which is an argument by tower rule, and takes use of independence between $Z$ and $Z'$:
    \begin{align*}
        \E[\mathbb I(\psi(\hat f(Z)))\le \psi'(\hat f(Z),Z')+\varepsilon] &= \E[\underbrace{\E[\mathbb I(\psi(\hat f(Z)))\le \psi'(\hat f(Z),Z')+\varepsilon \vert Z]}_{\ge 1-\delta_2}]\\
        &\ge 1-\delta_2
    \end{align*}
    Chaining the results together, and apply an union bound, it follows that w.p. $1-\delta_1-\delta_2$, \[\psi(\hat f(Z))\le \psi'(\hat f(Z),Z)+\varepsilon,\]
    in other words, $\PP(\tilde \cE)\ge 1-\delta_1-\delta_2$. 
    
\end{proof}

Before we start to prove \cref{thm:unif_L1}, we state an auxiliary lemma.
\begin{lemma}\label{lem:aux_lem}
    For all $a\in \R^n$ and $\sigma\in \mathrm{Rad}(n)$, w.p. $1-\delta$ it holds that
    \begin{equation*}
        \frac{1}{n}\langle a,\sigma\rangle\le \sqrt{\frac{2\|a\|_2^2 \log(1/\delta)}{n^2}}.
    \end{equation*}
\end{lemma}
\begin{proof}
    Recall that $\sigma_i$ the $i$-th component of $\sigma$, for $i\in [n]$, is subgaussian with parameter $1/2$. Then since $\sigma_i$'s are independent of each other, $\langle a, \sigma\rangle = \sum_{i=1}^n a_i\sigma_i$ is $\frac{1}{2}\sqrt{\sum_{i=1}^n a_i^2}$-subgaussian. By the property of a subgaussian random variable, it follows that
    \begin{align*}
        \PP(\langle a, \sigma\rangle \ge \varepsilon)\le \exp\left(-\frac{2\varepsilon^2}{0.5\sum_{i=1}^n a_i^2}\right)=\exp\left(-\frac{2\varepsilon^2}{0.5\|a\|_2^2}\right)
    \end{align*}
    By setting $\varepsilon=\sqrt{\frac{\|a\|_2^2\log(1/\delta)}{4}}$ we get a tighter bound.
\end{proof}
We now have the ingredients to prove a theorem. For convenience, let's state symmetrization lemma here without proof.
\begin{proposition}[Symmetrization Lemma]\label{thm:sym_lem}
    Let $\sigma\sim \mathrm{Rad}(n)$ be a sample of Rademachar distribution (the discrete uniform distribution over $\{\pm 1\}^n$) that is independent of $Z_{1:n}, Z_{1:n}'$. For $\cF\subseteq \R^{\cZ}$, functions $\psi:\cF\times \cZ^n\to \R$, $\tilde \psi:\cF\times \cZ^{2n}\to\R$ and $\varepsilon>0$, $0<\delta<1$, assume the following holds:
    \begin{enumerate}
       \item[(U)] w.p. $1-\delta$, for all $f\in \cF$, it follows that \[P_{\sigma,n}f\le \psi(f,Z_{1:n})+\varepsilon.\]
       \item[(Sub-additive)] For all $f\in \cF$, $z_{1:2n}\in \cZ^{2n}$, it follows that \[\psi(f,z_{1:n})+\psi(f,z_{n+1:2n})\le \tilde\psi(f,z_{1:2n}).\]
       \item[(S)] For all $f\in \cF$, $z_{1:2n}\in \cZ^{2n}$ and $\pi\in \mathrm{Perm}([2n])$, it follows that
       \[\tilde \psi(f,z_{1:2n})=\tilde \psi(f,z_{\pi(1:2n)})\]
       where $\mathrm{Perm}(A)=\{f:A\to A\vert f \text{ is a bijection}\}$ is the set of all the permutations on a finite set $A$ and $z_{\pi(1:2n)}=z_{\pi(1):\pi(2n)}$.
    \end{enumerate}
    Then w.p. $1-2\delta$, it holds that for all $f\in \cF$,
    \begin{equation*}
       P_n'f\le P_nf+\tilde\psi(f,Z_{1:n}Z_{1:n}')+2\varepsilon.
    \end{equation*}
 \end{proposition}

 \begin{proof}[proof of \cref{thm:unif_L1}]
    Let $\cF=\cG-\{0.5\}=\{g-0.5:g\in \cG\}$ where ``$-$'' is the Minkowski subtraction. Then by definition $\cF\subset [-0.5,0.5]$. Fix $\delta\in (0,1)$ and take $2n$ independent samples $(Z_{1:n},Z_{1:n}')\sim P^{\otimes n}$ and take $\sigma\sim \mathrm{Rad}(n)$ independent of $Z_{1:n}, Z_{1:n}'$. Let $\cF_\varepsilon(Z_{1:n})$ be $L_1(Z_{1:n})$ cover of $\cF$ such that $|F_\varepsilon(Z_{1:n})|\le N_1$. We now try to match with the conditions of \cref{thm:sym_lem}. By \cref{lem:aux_lem}, it follows that, for all $\delta_1\in (0,1)$, w.p. $1-\delta_1$, for all $f\in \cF_\varepsilon(Z_{1:n})$,
    \begin{equation*}
        P_{\sigma,n}f \le \sqrt{\frac{2\|f_n\|_2^2\log(N_1/\delta_1)}{n^2}}\le \sqrt{\frac{\log(N_1/\delta_1)}{2n}},
    \end{equation*}
    where $f_n=(f(Z_1),...,f(Z_n))$ and $\|f_n\|_2^2\le \frac{1}{4}\cdot n$.
 \end{proof}
 Fix $f\in \cF$, there exists $f'\in \Fc$ such that $P_n|f-f'|\le \varepsilon$. Now 
 \begin{align*}
    P_{\sigma,n}f&= P_{\sigma, n}f'+P_{\sigma,n}(f-f')\\
    &\le P_{\sigma,n}f'+P_n|f-f|\\
    &\le P_{\sigma,n}f'+\varepsilon\le \varepsilon+\sqrt{\frac{\log(N_1/\delta_1)}{2n}}.
 \end{align*}
 In the symmetrization lemma \cref{thm:sym_lem} we can choose $\psi=\tilde \psi=0$ and let $\varepsilon_{\text{sym}}=\varepsilon+\sqrt{\frac{\log(N_1/\delta)}{2n}}$. Then by applying \cref{thm:sym_lem}, w.p. $1-\delta_1$
 \[P_n'f\le P_nf + 2\left(\varepsilon+\sqrt{\frac{\log(N_1/\delta_1)}{2n}}\right).\]
 Since $\cF$ is just a shift of $\cG$, we shift it back
 \[P_n'g\le P_ng + 2\left(\varepsilon+\sqrt{\frac{\log(N_1/\delta_1)}{2n}}\right).\]
 Fix $g\in \cG$, by Hoeffding's inequality, for all $\delta_2\in (0,1)$, w.p. $1-\delta_2$,
 \begin{equation*}
    Pg\le P_n'g+\sqrt{\frac{\log(N_1/\delta_2)}{2n}}.
 \end{equation*}
 Matching the conditions of \cref{lem:psi}: $\psi(g)=Pg$, $\psi'(g,Z_{1:n}')=P_n'g+\sqrt{\frac{\log(1/\delta_2)}{2n}},\psi(g,Z)=P_ng+2\left(\varepsilon+\sqrt{\frac{\log(N_1/\delta_1)}{2n}}\right)+\sqrt{\frac{\log(1/\delta_2)}{2n}}$ and $\varepsilon_{lemma:10.2}=0$. Applying \cref{lem:psi}, we have that w.p. $1-\delta_1-\delta_2$,
 \begin{equation*}
    Pg \le P_ng +\sqrt{\frac{\log(1/\delta_2)}{2n}}+2\left(\varepsilon+\sqrt{\frac{\log(N_1/\delta_1)}{2n}}\right).
 \end{equation*}
Balancing $\delta_1+\delta_2=\delta$ where $\delta_1=(N_1\delta)/(2N_1+1)$ and $\delta_2=\delta-\delta_1$.
\end{document}
